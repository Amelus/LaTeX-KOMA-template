
\section{How to use this \LaTeX{} document template}

This \LaTeX{} document template from
\myLaT\footnote{\url{http://LaTeX.TUGraz.at}} is based on \myabk{KOMA}
script\footnote{\url{http://komascript.de/}}. It provides an easy to use and
easy to modify template. All settings are documented and many references to
additional information sources are given.

\subsection{How to compile this document}

\subsubsection{\textsc{GNU}/Linux, \textsc{OS~X}, \textsc{UNIX}, Cygwin}

If your system provides \myabk{GNU}
make\footnote{\url{https://secure.wikimedia.org/wikipedia/en/wiki/Make\_\%28software\%29}},
it is very easy to compile this document:

\begin{verbatim}
make pdf
\end{verbatim}

You can get a list of all other commands provided by the Makefile by invoking
\texttt{make help}.

\subsubsection{All other systems including Microsoft Windows}

If your system does not provide \myabk{GNU} make or you do not want to use \myabk{GNU} make,
you can compile this document using the usual method with pdf\LaTeX{}:

\begin{verbatim}
pdflatex main.tex
pdflatex main.tex
\end{verbatim}

If you are using \textsc{Bib}\TeX{} you have to add its commands such as:

\begin{verbatim}
pdflatex main.tex
bibtex main
pdflatex main.tex
pdflatex main.tex
\end{verbatim}

Addidional commands are required for packages like \texttt{makeindex} and so
forth.

\subsection{How to get rid of the template documentation}

The lines you are currently reading are being generated directly from the
included \TeX{} files of the preamble. You can remove them by simply removing
the line (or putting the comment character \texttt{\%} infront of it) 
\verb#
\section{How to use this \LaTeX{} document template}

This \LaTeX{} document template from
\myLaT\footnote{\url{http://LaTeX.TUGraz.at}} is based on \myabk{KOMA}
script\footnote{\url{http://komascript.de/}}. It provides an easy to use and
easy to modify template. All settings are documented and many references to
additional information sources are given.

\subsection{How to compile this document}

\subsubsection{\textsc{GNU}/Linux, \textsc{OS~X}, \textsc{UNIX}, Cygwin}

If your system provides \myabk{GNU}
make\footnote{\url{https://secure.wikimedia.org/wikipedia/en/wiki/Make\_\%28software\%29}},
it is very easy to compile this document:

\begin{verbatim}
make pdf
\end{verbatim}

You can get a list of all other commands provided by the Makefile by invoking
\texttt{make help}.

\subsubsection{All other systems including Microsoft Windows}

If your system does not provide \myabk{GNU} make or you do not want to use \myabk{GNU} make,
you can compile this document using the usual method with pdf\LaTeX{}:

\begin{verbatim}
pdflatex main.tex
pdflatex main.tex
\end{verbatim}

If you are using \textsc{Bib}\TeX{} you have to add its commands such as:

\begin{verbatim}
pdflatex main.tex
bibtex main
pdflatex main.tex
pdflatex main.tex
\end{verbatim}

Addidional commands are required for packages like \texttt{makeindex} and so
forth.

\subsection{How to get rid of the template documentation}

The lines you are currently reading are being generated directly from the
included \TeX{} files of the preamble. You can remove them by simply removing
the line (or putting the comment character \texttt{\%} infront of it) 
\verb#
\section{How to use this \LaTeX{} document template}

This \LaTeX{} document template from
\myLaT\footnote{\url{http://LaTeX.TUGraz.at}} is based on \myabk{KOMA}
script\footnote{\url{http://komascript.de/}}. It provides an easy to use and
easy to modify template. All settings are documented and many references to
additional information sources are given.

\subsection{How to compile this document}

\subsubsection{\textsc{GNU}/Linux, \textsc{OS~X}, \textsc{UNIX}, Cygwin}

If your system provides \myabk{GNU}
make\footnote{\url{https://secure.wikimedia.org/wikipedia/en/wiki/Make\_\%28software\%29}},
it is very easy to compile this document:

\begin{verbatim}
make pdf
\end{verbatim}

You can get a list of all other commands provided by the Makefile by invoking
\texttt{make help}.

\subsubsection{All other systems including Microsoft Windows}

If your system does not provide \myabk{GNU} make or you do not want to use \myabk{GNU} make,
you can compile this document using the usual method with pdf\LaTeX{}:

\begin{verbatim}
pdflatex main.tex
pdflatex main.tex
\end{verbatim}

If you are using \textsc{Bib}\TeX{} you have to add its commands such as:

\begin{verbatim}
pdflatex main.tex
bibtex main
pdflatex main.tex
pdflatex main.tex
\end{verbatim}

Addidional commands are required for packages like \texttt{makeindex} and so
forth.

\subsection{How to get rid of the template documentation}

The lines you are currently reading are being generated directly from the
included \TeX{} files of the preamble. You can remove them by simply removing
the line (or putting the comment character \texttt{\%} infront of it) 
\verb#
\section{How to use this \LaTeX{} document template}

This \LaTeX{} document template from
\myLaT\footnote{\url{http://LaTeX.TUGraz.at}} is based on \myabk{KOMA}
script\footnote{\url{http://komascript.de/}}. It provides an easy to use and
easy to modify template. All settings are documented and many references to
additional information sources are given.

\subsection{How to compile this document}

\subsubsection{\textsc{GNU}/Linux, \textsc{OS~X}, \textsc{UNIX}, Cygwin}

If your system provides \myabk{GNU}
make\footnote{\url{https://secure.wikimedia.org/wikipedia/en/wiki/Make\_\%28software\%29}},
it is very easy to compile this document:

\begin{verbatim}
make pdf
\end{verbatim}

You can get a list of all other commands provided by the Makefile by invoking
\texttt{make help}.

\subsubsection{All other systems including Microsoft Windows}

If your system does not provide \myabk{GNU} make or you do not want to use \myabk{GNU} make,
you can compile this document using the usual method with pdf\LaTeX{}:

\begin{verbatim}
pdflatex main.tex
pdflatex main.tex
\end{verbatim}

If you are using \textsc{Bib}\TeX{} you have to add its commands such as:

\begin{verbatim}
pdflatex main.tex
bibtex main
pdflatex main.tex
pdflatex main.tex
\end{verbatim}

Addidional commands are required for packages like \texttt{makeindex} and so
forth.

\subsection{How to get rid of the template documentation}

The lines you are currently reading are being generated directly from the
included \TeX{} files of the preamble. You can remove them by simply removing
the line (or putting the comment character \texttt{\%} infront of it) 
\verb#\input{template_documentation}# in \texttt{main.tex}.

\subsection{What about modifying the template?}

This template provides an easy to start \LaTeX{} document template with sound
default settings. You can modify each setting any time. It is recommended that
you are familiar with the documentation of the command whose settings you want
to modify.

The following sections describe the settings and commands of this template and
gives a short overview of its features.
%doc%
\section{\texttt{preamble.tex} --- Main preamble file}
%doc%
In file \verb#preamble/preamble.tex# you will find the basic
definitions related to your document. This template uses the \myabk{KOMA} script
extension package of \LaTeX{}.

There are comments added to the \verb#\documentclass{}# definitions. Please
refer to the great documentation of \myabk{KOMA}\footnote{\texttt{scrguide.pdf} for
German users} for further details.

\paragraph{What should I do with this file?} For standard purposes you might
use the default values it provides. You must not remove its \texttt{include} command
in \texttt{main.tex} since it contains important definitions. This file contains
settings which are documented well an can be modified according to your needs.
It is recommended that you fully understand each setting you modify in order to
get a good document result.


\subsection{UTF8 as input charset}

You are able and should use \myabk{UTF8} character settings for writing these \TeX{}-files.


\subsection{Language settings}

The default setting of the language is English. Please change settings for
additional or alternative languages used.


\subsection{Headers and footers}

Since this template is based on \myabk{KOMA} script it uses its great \texttt{scrpage2}
package for defining header and footer information. Please refer to the \myabk{KOMA}
script documentation how to use this package.


\subsection{Miscellaneous packages} \label{subsec:miscpackages}

There are several packages included by default. You might want to activate or
deactivate them according to your requirements:

\begin{enumerate}
\item[\texttt{\href{https://secure.wikimedia.org/wikibooks/en/wiki/LaTeX/Formatting\#Other\_symbols}{%%
pifont%%
}}] 
For additional special characters available by \verb#\ding{}#
\item[\texttt{\href{http://ctan.org/pkg/ifthen}{%%
ifthen%%
}}] 
For using if/then/else statements for example in macros
\item[\texttt{\href{http://www.ctan.org/tex-archive/fonts/eurosym}{%%
eurosym%%
}}] 
Using the character for Euro with \verb#\officialeuro{}#
\item[\texttt{\href{http://www.ctan.org/tex-archive/help/Catalogue/entries/xspace.html}{%%
xspace%%
}}] 
This package is required for intelligent spacing after commands
\item[\texttt{\href{https://secure.wikimedia.org/wikibooks/en/wiki/LaTeX/Colors}{%%
color%%
}}] 
This package defines basic colors
\end{enumerate}
%doc%
\section{\texttt{typographic\_settings.tex} --- Typographic finetuning}
%doc%
The settings of file \verb#preamble/typographic_settings.tex# contain
typographic finetuning related to things mentioned in literature.  The
settings in this file relates to personal taste and most of all typographic
experience. 

\paragraph{What should I do with this file?} You might as well skip the whole
file by excluding the \verb#\input{preamble/typographic_settings.tex}# command
in \texttt{main.tex}.  For standard usage it is recommended to stay with the
default settings.


\subsection{References related to typographic settings}

\begin{thebibliography}{9}
\bibitem[Bringhurst1993]{Bringhurst1993}
    \textbf{Robert Bringhurst}\\
    \textit{The Elements of Typographic Style}\\
    paperback, first edition, 1993
\bibitem[Eijkhout2008]{Eijkhout2008}
    \textbf{Victor Eijkhout}\\
    \textit{\TeX{} by Topic, a \TeX{}nician's Reference}\\
    document revision 1.2, may 2008\\
    \url{http://www.eijkhout.net/texbytopic/texbytopic.html}
\end{thebibliography}
%doc%
\subsection{French spacing}
%doc%
\paragraph{Why?} \cite[p 28, p 30]{Bringhurst1993}: `2.1.4 Use a single word space between sentences.'
%doc%
\paragraph{How?} \cite[p 185]{Eijkhout2008}:\\
\verb#\frenchspacing  %% Macro to switch off extra space after punctuation.# \\
%doc%
Note: This setting might be default for \myabk{KOMA} script.
%doc%
%doc%
\subsection{Text figures}

\ldots also called old style numbers. 
(German: Mediävalziffern\footnote{\url{https://secure.wikimedia.org/wikibooks/de/wiki/LaTeX-W\%C3\%B6rterbuch:\_Medi\%C3\%A4valziffern}})
\paragraph{Why?} \cite[p 44f]{Bringhurst1993}: 
\begin{quote}
`3.2.1 If the font includes both text figures and titling figures, use
 titling figures only with full caps, and text figures in all other
 circumstances.'
\end{quote}

\paragraph{How?} 
Quoted from Wikibooks\footnote{\url{https://secure.wikimedia.org/wikibooks/en/wiki/LaTeX/Formatting\#Text\_figures\_.28.22old\_style.22\_numerals.29}}:
\begin{quote}
Some fonts do not have text figures built in; the textcomp package attempts to
remedy this by effectively generating text figures from the currently-selected
font. Put \verb#\usepackage{textcomp}# in your preamble. textcomp also allows you to
use decimal points, properly formatted dollar signs, etc. within
\verb#\oldstylenums{}#.
\end{quote}
\ldots but proposed \LaTeX{} method does not work out well. Instead use:\\
\verb#\usepackage{hfoldsty}#  (enables text figures using additional font) or \\
\verb#\usepackage[sc,osf]{mathpazo}# (switches to Palatino font with small caps and old style figures enabled).
%doc%

\subsection{Abbrevations using \textsc{small caps}}

\paragraph{Why?} \cite[p 45f]{Bringhurst1993}: `3.2.2 For abbrevations and
acronyms in the midst of normal text, use spaced small caps.'

\paragraph{How?} Using the predefined macro \verb#\myabk{}# for things like
\myabk{UNO} or \myabk{UNESCO} using \verb#\myabk{UNO}# or \verb#\myabk{UNESCO}#.


\subsection{Colorized headings and links}

This document template is able to generate an output that uses colorized
headings, captions, page numbers, and links. The color named `DispositionColor'
used in this document is defined near the definition of package \texttt{color}
in the preamble (see section~\ref{subsec:miscpackages}). The changes required
for headings, page numbers, and captions are defined here.

Settings for colored links are handled by the definitions of the
\texttt{hyperref} package (see section~\ref{sec:pdf}).

%doc%
\section{\texttt{pdf\_settings.tex} --- Settings related to PDF output}
\label{sec:pdf}

The file \verb#preamble/pdf_settings.tex# basically contains the definitions for
the \href{http://tug.org/applications/hyperref/}{\texttt{hyperref} package}
including the
\href{http://www.ctan.org/tex-archive/macros/latex/required/graphics/}{\texttt{graphicx}
package}. Since these settings should be the last things of any \LaTeX{}
preamble, they got their own \TeX{} file which is included in \texttt{main.tex}.

\paragraph{What should I do with this file?} The settings in this file are
important for \myabk{PDF} output and including graphics. Do not exclude the
related \texttt{input} command in \texttt{main.tex}. But you might want to
modify some settings after you read the
\href{http://tug.org/applications/hyperref/}{documentation of the \texttt{hyperref} package}.

# in \texttt{main.tex}.

\subsection{What about modifying the template?}

This template provides an easy to start \LaTeX{} document template with sound
default settings. You can modify each setting any time. It is recommended that
you are familiar with the documentation of the command whose settings you want
to modify.

The following sections describe the settings and commands of this template and
gives a short overview of its features.
%doc%
\section{\texttt{preamble.tex} --- Main preamble file}
%doc%
In file \verb#preamble/preamble.tex# you will find the basic
definitions related to your document. This template uses the \myabk{KOMA} script
extension package of \LaTeX{}.

There are comments added to the \verb#\documentclass{}# definitions. Please
refer to the great documentation of \myabk{KOMA}\footnote{\texttt{scrguide.pdf} for
German users} for further details.

\paragraph{What should I do with this file?} For standard purposes you might
use the default values it provides. You must not remove its \texttt{include} command
in \texttt{main.tex} since it contains important definitions. This file contains
settings which are documented well an can be modified according to your needs.
It is recommended that you fully understand each setting you modify in order to
get a good document result.


\subsection{UTF8 as input charset}

You are able and should use \myabk{UTF8} character settings for writing these \TeX{}-files.


\subsection{Language settings}

The default setting of the language is English. Please change settings for
additional or alternative languages used.


\subsection{Headers and footers}

Since this template is based on \myabk{KOMA} script it uses its great \texttt{scrpage2}
package for defining header and footer information. Please refer to the \myabk{KOMA}
script documentation how to use this package.


\subsection{Miscellaneous packages} \label{subsec:miscpackages}

There are several packages included by default. You might want to activate or
deactivate them according to your requirements:

\begin{enumerate}
\item[\texttt{\href{https://secure.wikimedia.org/wikibooks/en/wiki/LaTeX/Formatting\#Other\_symbols}{%%
pifont%%
}}] 
For additional special characters available by \verb#\ding{}#
\item[\texttt{\href{http://ctan.org/pkg/ifthen}{%%
ifthen%%
}}] 
For using if/then/else statements for example in macros
\item[\texttt{\href{http://www.ctan.org/tex-archive/fonts/eurosym}{%%
eurosym%%
}}] 
Using the character for Euro with \verb#\officialeuro{}#
\item[\texttt{\href{http://www.ctan.org/tex-archive/help/Catalogue/entries/xspace.html}{%%
xspace%%
}}] 
This package is required for intelligent spacing after commands
\item[\texttt{\href{https://secure.wikimedia.org/wikibooks/en/wiki/LaTeX/Colors}{%%
color%%
}}] 
This package defines basic colors
\end{enumerate}
%doc%
\section{\texttt{typographic\_settings.tex} --- Typographic finetuning}
%doc%
The settings of file \verb#preamble/typographic_settings.tex# contain
typographic finetuning related to things mentioned in literature.  The
settings in this file relates to personal taste and most of all typographic
experience. 

\paragraph{What should I do with this file?} You might as well skip the whole
file by excluding the \verb#%%%% Time-stamp: <2011-07-22 15:45:04 vk>
%%%% === Disclaimer: =======================================================
%% created by
%%
%%      Karl Voit
%%
%% using grml GNU/Linux, vim & LaTeX 2e
%%
%doc%
%doc% \section{\texttt{typographic\_settings.tex} --- Typographic finetuning}
%doc%
%doc% The settings of file \verb#preamble/typographic_settings.tex# contain
%doc% typographic finetuning related to things mentioned in literature.  The
%doc% settings in this file relates to personal taste and most of all typographic
%doc% experience. 
%doc% 
%doc% \paragraph{What should I do with this file?} You might as well skip the whole
%doc% file by excluding the \verb#\input{preamble/typographic_settings.tex}# command
%doc% in \texttt{main.tex}.  For standard usage it is recommended to stay with the
%doc% default settings.
%doc% 
%doc% 
%doc% \subsection{References related to typographic settings}
%doc% 
%doc% \begin{thebibliography}{9}
%doc% \bibitem[Bringhurst1993]{Bringhurst1993}
%doc%     \textbf{Robert Bringhurst}\\
%doc%     \textit{The Elements of Typographic Style}\\
%doc%     paperback, first edition, 1993
%doc% \bibitem[Eijkhout2008]{Eijkhout2008}
%doc%     \textbf{Victor Eijkhout}\\
%doc%     \textit{\TeX{} by Topic, a \TeX{}nician's Reference}\\
%doc%     document revision 1.2, may 2008\\
%doc%     \url{http://www.eijkhout.net/texbytopic/texbytopic.html}
%doc% \end{thebibliography}
%% ========================================================================


%doc%
%doc% \subsection{French spacing}
%doc%
%doc% \paragraph{Why?} \cite[p 28, p 30]{Bringhurst1993}: `2.1.4 Use a single word space between sentences.'
%doc%
%doc% \paragraph{How?} \cite[p 185]{Eijkhout2008}:\\
%doc% \verb#\frenchspacing  %% Macro to switch off extra space after punctuation.# \\
\frenchspacing  %% Macro to switch off extra space after punctuation.
%doc%
%doc% Note: This setting might be default for \myabk{KOMA} script.
%doc%


%doc%
%doc% \subsection{Text figures}
%doc% 
%doc% \ldots also called old style numbers. 
%doc% (German: Mediävalziffern\footnote{\url{https://secure.wikimedia.org/wikibooks/de/wiki/LaTeX-W\%C3\%B6rterbuch:\_Medi\%C3\%A4valziffern}})
%doc% \paragraph{Why?} \cite[p 44f]{Bringhurst1993}: 
%doc% \begin{quote}
%doc% `3.2.1 If the font includes both text figures and titling figures, use
%doc%  titling figures only with full caps, and text figures in all other
%doc%  circumstances.'
%doc% \end{quote}
%doc% 
%doc% \paragraph{How?} 
%doc% Quoted from Wikibooks\footnote{\url{https://secure.wikimedia.org/wikibooks/en/wiki/LaTeX/Formatting\#Text\_figures\_.28.22old\_style.22\_numerals.29}}:
%doc% \begin{quote}
%doc% Some fonts do not have text figures built in; the textcomp package attempts to
%doc% remedy this by effectively generating text figures from the currently-selected
%doc% font. Put \verb#\usepackage{textcomp}# in your preamble. textcomp also allows you to
%doc% use decimal points, properly formatted dollar signs, etc. within
%doc% \verb#\oldstylenums{}#.
%doc% \end{quote}
%doc% \ldots but proposed \LaTeX{} method does not work out well. Instead use:\\
%doc% \verb#\usepackage{hfoldsty}#  (enables text figures using additional font) or \\
%doc% \verb#\usepackage[sc,osf]{mathpazo}# (switches to Palatino font with small caps and old style figures enabled).
%doc%
%\usepackage{hfoldsty}  %% enables text figures using additional font
%% ... OR use ...
\usepackage[sc,osf]{mathpazo} %% switches to Palatino with small caps and old style figures


%doc% 
%doc% \subsection{Abbrevations using \textsc{small caps}}
%doc% 
%doc% \paragraph{Why?} \cite[p 45f]{Bringhurst1993}: `3.2.2 For abbrevations and
%doc% acronyms in the midst of normal text, use spaced small caps.'
%doc% 
%doc% \paragraph{How?} Using the predefined macro \verb#\myabk{}# for things like
%doc% \myabk{UNO} or \myabk{UNESCO} using \verb#\myabk{UNO}# or \verb#\myabk{UNESCO}#.
%doc% 
\newcommand{\myabk}[1]{%%  abbrevations using small caps
\textsc{\lowercase{#1}}%%
}


%doc% 
%doc% \subsection{Colorized headings and links}
%doc% 
%doc% This document template is able to generate an output that uses colorized
%doc% headings, captions, page numbers, and links. The color named `DispositionColor'
%doc% used in this document is defined near the definition of package \texttt{color}
%doc% in the preamble (see section~\ref{subsec:miscpackages}). The changes required
%doc% for headings, page numbers, and captions are defined here.
%doc% 
%doc% Settings for colored links are handled by the definitions of the
%doc% \texttt{hyperref} package (see section~\ref{sec:pdf}).
%doc% 
\setheadsepline{.4pt}[\color{DispositionColor}]
\renewcommand{\headfont}{\normalfont\sffamily\color{DispositionColor}}
\renewcommand{\pnumfont}{\normalfont\sffamily\color{DispositionColor}}
\addtokomafont{disposition}{\color{DispositionColor}}
\addtokomafont{caption}{\color{DispositionColor}\footnotesize}
\addtokomafont{captionlabel}{\color{DispositionColor}}

%%%% END
%%% Local Variables:
%%% TeX-master: "../main"
%%% End:
%% vim:foldmethod=expr
%% vim:fde=getline(v\:lnum)=~'^%%%%'?0\:getline(v\:lnum)=~'^%doc.*\ .\\%(sub\\)\\?section{.\\+'?'>1'\:'1':
# command
in \texttt{main.tex}.  For standard usage it is recommended to stay with the
default settings.


\subsection{References related to typographic settings}

\begin{thebibliography}{9}
\bibitem[Bringhurst1993]{Bringhurst1993}
    \textbf{Robert Bringhurst}\\
    \textit{The Elements of Typographic Style}\\
    paperback, first edition, 1993
\bibitem[Eijkhout2008]{Eijkhout2008}
    \textbf{Victor Eijkhout}\\
    \textit{\TeX{} by Topic, a \TeX{}nician's Reference}\\
    document revision 1.2, may 2008\\
    \url{http://www.eijkhout.net/texbytopic/texbytopic.html}
\end{thebibliography}
%doc%
\subsection{French spacing}
%doc%
\paragraph{Why?} \cite[p 28, p 30]{Bringhurst1993}: `2.1.4 Use a single word space between sentences.'
%doc%
\paragraph{How?} \cite[p 185]{Eijkhout2008}:\\
\verb#\frenchspacing  %% Macro to switch off extra space after punctuation.# \\
%doc%
Note: This setting might be default for \myabk{KOMA} script.
%doc%
%doc%
\subsection{Text figures}

\ldots also called old style numbers. 
(German: Mediävalziffern\footnote{\url{https://secure.wikimedia.org/wikibooks/de/wiki/LaTeX-W\%C3\%B6rterbuch:\_Medi\%C3\%A4valziffern}})
\paragraph{Why?} \cite[p 44f]{Bringhurst1993}: 
\begin{quote}
`3.2.1 If the font includes both text figures and titling figures, use
 titling figures only with full caps, and text figures in all other
 circumstances.'
\end{quote}

\paragraph{How?} 
Quoted from Wikibooks\footnote{\url{https://secure.wikimedia.org/wikibooks/en/wiki/LaTeX/Formatting\#Text\_figures\_.28.22old\_style.22\_numerals.29}}:
\begin{quote}
Some fonts do not have text figures built in; the textcomp package attempts to
remedy this by effectively generating text figures from the currently-selected
font. Put \verb#\usepackage{textcomp}# in your preamble. textcomp also allows you to
use decimal points, properly formatted dollar signs, etc. within
\verb#\oldstylenums{}#.
\end{quote}
\ldots but proposed \LaTeX{} method does not work out well. Instead use:\\
\verb#\usepackage{hfoldsty}#  (enables text figures using additional font) or \\
\verb#\usepackage[sc,osf]{mathpazo}# (switches to Palatino font with small caps and old style figures enabled).
%doc%

\subsection{Abbrevations using \textsc{small caps}}

\paragraph{Why?} \cite[p 45f]{Bringhurst1993}: `3.2.2 For abbrevations and
acronyms in the midst of normal text, use spaced small caps.'

\paragraph{How?} Using the predefined macro \verb#\myabk{}# for things like
\myabk{UNO} or \myabk{UNESCO} using \verb#\myabk{UNO}# or \verb#\myabk{UNESCO}#.


\subsection{Colorized headings and links}

This document template is able to generate an output that uses colorized
headings, captions, page numbers, and links. The color named `DispositionColor'
used in this document is defined near the definition of package \texttt{color}
in the preamble (see section~\ref{subsec:miscpackages}). The changes required
for headings, page numbers, and captions are defined here.

Settings for colored links are handled by the definitions of the
\texttt{hyperref} package (see section~\ref{sec:pdf}).

%doc%
\section{\texttt{pdf\_settings.tex} --- Settings related to PDF output}
\label{sec:pdf}

The file \verb#preamble/pdf_settings.tex# basically contains the definitions for
the \href{http://tug.org/applications/hyperref/}{\texttt{hyperref} package}
including the
\href{http://www.ctan.org/tex-archive/macros/latex/required/graphics/}{\texttt{graphicx}
package}. Since these settings should be the last things of any \LaTeX{}
preamble, they got their own \TeX{} file which is included in \texttt{main.tex}.

\paragraph{What should I do with this file?} The settings in this file are
important for \myabk{PDF} output and including graphics. Do not exclude the
related \texttt{input} command in \texttt{main.tex}. But you might want to
modify some settings after you read the
\href{http://tug.org/applications/hyperref/}{documentation of the \texttt{hyperref} package}.

# in \texttt{main.tex}.

\subsection{What about modifying the template?}

This template provides an easy to start \LaTeX{} document template with sound
default settings. You can modify each setting any time. It is recommended that
you are familiar with the documentation of the command whose settings you want
to modify.

The following sections describe the settings and commands of this template and
gives a short overview of its features.
%doc%
\section{\texttt{preamble.tex} --- Main preamble file}
%doc%
In file \verb#preamble/preamble.tex# you will find the basic
definitions related to your document. This template uses the \myabk{KOMA} script
extension package of \LaTeX{}.

There are comments added to the \verb#\documentclass{}# definitions. Please
refer to the great documentation of \myabk{KOMA}\footnote{\texttt{scrguide.pdf} for
German users} for further details.

\paragraph{What should I do with this file?} For standard purposes you might
use the default values it provides. You must not remove its \texttt{include} command
in \texttt{main.tex} since it contains important definitions. This file contains
settings which are documented well an can be modified according to your needs.
It is recommended that you fully understand each setting you modify in order to
get a good document result.


\subsection{UTF8 as input charset}

You are able and should use \myabk{UTF8} character settings for writing these \TeX{}-files.


\subsection{Language settings}

The default setting of the language is English. Please change settings for
additional or alternative languages used.


\subsection{Headers and footers}

Since this template is based on \myabk{KOMA} script it uses its great \texttt{scrpage2}
package for defining header and footer information. Please refer to the \myabk{KOMA}
script documentation how to use this package.


\subsection{Miscellaneous packages} \label{subsec:miscpackages}

There are several packages included by default. You might want to activate or
deactivate them according to your requirements:

\begin{enumerate}
\item[\texttt{\href{https://secure.wikimedia.org/wikibooks/en/wiki/LaTeX/Formatting\#Other\_symbols}{%%
pifont%%
}}] 
For additional special characters available by \verb#\ding{}#
\item[\texttt{\href{http://ctan.org/pkg/ifthen}{%%
ifthen%%
}}] 
For using if/then/else statements for example in macros
\item[\texttt{\href{http://www.ctan.org/tex-archive/fonts/eurosym}{%%
eurosym%%
}}] 
Using the character for Euro with \verb#\officialeuro{}#
\item[\texttt{\href{http://www.ctan.org/tex-archive/help/Catalogue/entries/xspace.html}{%%
xspace%%
}}] 
This package is required for intelligent spacing after commands
\item[\texttt{\href{https://secure.wikimedia.org/wikibooks/en/wiki/LaTeX/Colors}{%%
color%%
}}] 
This package defines basic colors
\end{enumerate}
%doc%
\section{\texttt{typographic\_settings.tex} --- Typographic finetuning}
%doc%
The settings of file \verb#preamble/typographic_settings.tex# contain
typographic finetuning related to things mentioned in literature.  The
settings in this file relates to personal taste and most of all typographic
experience. 

\paragraph{What should I do with this file?} You might as well skip the whole
file by excluding the \verb#%%%% Time-stamp: <2011-07-22 15:45:04 vk>
%%%% === Disclaimer: =======================================================
%% created by
%%
%%      Karl Voit
%%
%% using grml GNU/Linux, vim & LaTeX 2e
%%
%doc%
%doc% \section{\texttt{typographic\_settings.tex} --- Typographic finetuning}
%doc%
%doc% The settings of file \verb#preamble/typographic_settings.tex# contain
%doc% typographic finetuning related to things mentioned in literature.  The
%doc% settings in this file relates to personal taste and most of all typographic
%doc% experience. 
%doc% 
%doc% \paragraph{What should I do with this file?} You might as well skip the whole
%doc% file by excluding the \verb#%%%% Time-stamp: <2011-07-22 15:45:04 vk>
%%%% === Disclaimer: =======================================================
%% created by
%%
%%      Karl Voit
%%
%% using grml GNU/Linux, vim & LaTeX 2e
%%
%doc%
%doc% \section{\texttt{typographic\_settings.tex} --- Typographic finetuning}
%doc%
%doc% The settings of file \verb#preamble/typographic_settings.tex# contain
%doc% typographic finetuning related to things mentioned in literature.  The
%doc% settings in this file relates to personal taste and most of all typographic
%doc% experience. 
%doc% 
%doc% \paragraph{What should I do with this file?} You might as well skip the whole
%doc% file by excluding the \verb#\input{preamble/typographic_settings.tex}# command
%doc% in \texttt{main.tex}.  For standard usage it is recommended to stay with the
%doc% default settings.
%doc% 
%doc% 
%doc% \subsection{References related to typographic settings}
%doc% 
%doc% \begin{thebibliography}{9}
%doc% \bibitem[Bringhurst1993]{Bringhurst1993}
%doc%     \textbf{Robert Bringhurst}\\
%doc%     \textit{The Elements of Typographic Style}\\
%doc%     paperback, first edition, 1993
%doc% \bibitem[Eijkhout2008]{Eijkhout2008}
%doc%     \textbf{Victor Eijkhout}\\
%doc%     \textit{\TeX{} by Topic, a \TeX{}nician's Reference}\\
%doc%     document revision 1.2, may 2008\\
%doc%     \url{http://www.eijkhout.net/texbytopic/texbytopic.html}
%doc% \end{thebibliography}
%% ========================================================================


%doc%
%doc% \subsection{French spacing}
%doc%
%doc% \paragraph{Why?} \cite[p 28, p 30]{Bringhurst1993}: `2.1.4 Use a single word space between sentences.'
%doc%
%doc% \paragraph{How?} \cite[p 185]{Eijkhout2008}:\\
%doc% \verb#\frenchspacing  %% Macro to switch off extra space after punctuation.# \\
\frenchspacing  %% Macro to switch off extra space after punctuation.
%doc%
%doc% Note: This setting might be default for \myabk{KOMA} script.
%doc%


%doc%
%doc% \subsection{Text figures}
%doc% 
%doc% \ldots also called old style numbers. 
%doc% (German: Mediävalziffern\footnote{\url{https://secure.wikimedia.org/wikibooks/de/wiki/LaTeX-W\%C3\%B6rterbuch:\_Medi\%C3\%A4valziffern}})
%doc% \paragraph{Why?} \cite[p 44f]{Bringhurst1993}: 
%doc% \begin{quote}
%doc% `3.2.1 If the font includes both text figures and titling figures, use
%doc%  titling figures only with full caps, and text figures in all other
%doc%  circumstances.'
%doc% \end{quote}
%doc% 
%doc% \paragraph{How?} 
%doc% Quoted from Wikibooks\footnote{\url{https://secure.wikimedia.org/wikibooks/en/wiki/LaTeX/Formatting\#Text\_figures\_.28.22old\_style.22\_numerals.29}}:
%doc% \begin{quote}
%doc% Some fonts do not have text figures built in; the textcomp package attempts to
%doc% remedy this by effectively generating text figures from the currently-selected
%doc% font. Put \verb#\usepackage{textcomp}# in your preamble. textcomp also allows you to
%doc% use decimal points, properly formatted dollar signs, etc. within
%doc% \verb#\oldstylenums{}#.
%doc% \end{quote}
%doc% \ldots but proposed \LaTeX{} method does not work out well. Instead use:\\
%doc% \verb#\usepackage{hfoldsty}#  (enables text figures using additional font) or \\
%doc% \verb#\usepackage[sc,osf]{mathpazo}# (switches to Palatino font with small caps and old style figures enabled).
%doc%
%\usepackage{hfoldsty}  %% enables text figures using additional font
%% ... OR use ...
\usepackage[sc,osf]{mathpazo} %% switches to Palatino with small caps and old style figures


%doc% 
%doc% \subsection{Abbrevations using \textsc{small caps}}
%doc% 
%doc% \paragraph{Why?} \cite[p 45f]{Bringhurst1993}: `3.2.2 For abbrevations and
%doc% acronyms in the midst of normal text, use spaced small caps.'
%doc% 
%doc% \paragraph{How?} Using the predefined macro \verb#\myabk{}# for things like
%doc% \myabk{UNO} or \myabk{UNESCO} using \verb#\myabk{UNO}# or \verb#\myabk{UNESCO}#.
%doc% 
\newcommand{\myabk}[1]{%%  abbrevations using small caps
\textsc{\lowercase{#1}}%%
}


%doc% 
%doc% \subsection{Colorized headings and links}
%doc% 
%doc% This document template is able to generate an output that uses colorized
%doc% headings, captions, page numbers, and links. The color named `DispositionColor'
%doc% used in this document is defined near the definition of package \texttt{color}
%doc% in the preamble (see section~\ref{subsec:miscpackages}). The changes required
%doc% for headings, page numbers, and captions are defined here.
%doc% 
%doc% Settings for colored links are handled by the definitions of the
%doc% \texttt{hyperref} package (see section~\ref{sec:pdf}).
%doc% 
\setheadsepline{.4pt}[\color{DispositionColor}]
\renewcommand{\headfont}{\normalfont\sffamily\color{DispositionColor}}
\renewcommand{\pnumfont}{\normalfont\sffamily\color{DispositionColor}}
\addtokomafont{disposition}{\color{DispositionColor}}
\addtokomafont{caption}{\color{DispositionColor}\footnotesize}
\addtokomafont{captionlabel}{\color{DispositionColor}}

%%%% END
%%% Local Variables:
%%% TeX-master: "../main"
%%% End:
%% vim:foldmethod=expr
%% vim:fde=getline(v\:lnum)=~'^%%%%'?0\:getline(v\:lnum)=~'^%doc.*\ .\\%(sub\\)\\?section{.\\+'?'>1'\:'1':
# command
%doc% in \texttt{main.tex}.  For standard usage it is recommended to stay with the
%doc% default settings.
%doc% 
%doc% 
%doc% \subsection{References related to typographic settings}
%doc% 
%doc% \begin{thebibliography}{9}
%doc% \bibitem[Bringhurst1993]{Bringhurst1993}
%doc%     \textbf{Robert Bringhurst}\\
%doc%     \textit{The Elements of Typographic Style}\\
%doc%     paperback, first edition, 1993
%doc% \bibitem[Eijkhout2008]{Eijkhout2008}
%doc%     \textbf{Victor Eijkhout}\\
%doc%     \textit{\TeX{} by Topic, a \TeX{}nician's Reference}\\
%doc%     document revision 1.2, may 2008\\
%doc%     \url{http://www.eijkhout.net/texbytopic/texbytopic.html}
%doc% \end{thebibliography}
%% ========================================================================


%doc%
%doc% \subsection{French spacing}
%doc%
%doc% \paragraph{Why?} \cite[p 28, p 30]{Bringhurst1993}: `2.1.4 Use a single word space between sentences.'
%doc%
%doc% \paragraph{How?} \cite[p 185]{Eijkhout2008}:\\
%doc% \verb#\frenchspacing  %% Macro to switch off extra space after punctuation.# \\
\frenchspacing  %% Macro to switch off extra space after punctuation.
%doc%
%doc% Note: This setting might be default for \myabk{KOMA} script.
%doc%


%doc%
%doc% \subsection{Text figures}
%doc% 
%doc% \ldots also called old style numbers. 
%doc% (German: Mediävalziffern\footnote{\url{https://secure.wikimedia.org/wikibooks/de/wiki/LaTeX-W\%C3\%B6rterbuch:\_Medi\%C3\%A4valziffern}})
%doc% \paragraph{Why?} \cite[p 44f]{Bringhurst1993}: 
%doc% \begin{quote}
%doc% `3.2.1 If the font includes both text figures and titling figures, use
%doc%  titling figures only with full caps, and text figures in all other
%doc%  circumstances.'
%doc% \end{quote}
%doc% 
%doc% \paragraph{How?} 
%doc% Quoted from Wikibooks\footnote{\url{https://secure.wikimedia.org/wikibooks/en/wiki/LaTeX/Formatting\#Text\_figures\_.28.22old\_style.22\_numerals.29}}:
%doc% \begin{quote}
%doc% Some fonts do not have text figures built in; the textcomp package attempts to
%doc% remedy this by effectively generating text figures from the currently-selected
%doc% font. Put \verb#\usepackage{textcomp}# in your preamble. textcomp also allows you to
%doc% use decimal points, properly formatted dollar signs, etc. within
%doc% \verb#\oldstylenums{}#.
%doc% \end{quote}
%doc% \ldots but proposed \LaTeX{} method does not work out well. Instead use:\\
%doc% \verb#\usepackage{hfoldsty}#  (enables text figures using additional font) or \\
%doc% \verb#\usepackage[sc,osf]{mathpazo}# (switches to Palatino font with small caps and old style figures enabled).
%doc%
%\usepackage{hfoldsty}  %% enables text figures using additional font
%% ... OR use ...
\usepackage[sc,osf]{mathpazo} %% switches to Palatino with small caps and old style figures


%doc% 
%doc% \subsection{Abbrevations using \textsc{small caps}}
%doc% 
%doc% \paragraph{Why?} \cite[p 45f]{Bringhurst1993}: `3.2.2 For abbrevations and
%doc% acronyms in the midst of normal text, use spaced small caps.'
%doc% 
%doc% \paragraph{How?} Using the predefined macro \verb#\myabk{}# for things like
%doc% \myabk{UNO} or \myabk{UNESCO} using \verb#\myabk{UNO}# or \verb#\myabk{UNESCO}#.
%doc% 
\newcommand{\myabk}[1]{%%  abbrevations using small caps
\textsc{\lowercase{#1}}%%
}


%doc% 
%doc% \subsection{Colorized headings and links}
%doc% 
%doc% This document template is able to generate an output that uses colorized
%doc% headings, captions, page numbers, and links. The color named `DispositionColor'
%doc% used in this document is defined near the definition of package \texttt{color}
%doc% in the preamble (see section~\ref{subsec:miscpackages}). The changes required
%doc% for headings, page numbers, and captions are defined here.
%doc% 
%doc% Settings for colored links are handled by the definitions of the
%doc% \texttt{hyperref} package (see section~\ref{sec:pdf}).
%doc% 
\setheadsepline{.4pt}[\color{DispositionColor}]
\renewcommand{\headfont}{\normalfont\sffamily\color{DispositionColor}}
\renewcommand{\pnumfont}{\normalfont\sffamily\color{DispositionColor}}
\addtokomafont{disposition}{\color{DispositionColor}}
\addtokomafont{caption}{\color{DispositionColor}\footnotesize}
\addtokomafont{captionlabel}{\color{DispositionColor}}

%%%% END
%%% Local Variables:
%%% TeX-master: "../main"
%%% End:
%% vim:foldmethod=expr
%% vim:fde=getline(v\:lnum)=~'^%%%%'?0\:getline(v\:lnum)=~'^%doc.*\ .\\%(sub\\)\\?section{.\\+'?'>1'\:'1':
# command
in \texttt{main.tex}.  For standard usage it is recommended to stay with the
default settings.


\subsection{References related to typographic settings}

\begin{thebibliography}{9}
\bibitem[Bringhurst1993]{Bringhurst1993}
    \textbf{Robert Bringhurst}\\
    \textit{The Elements of Typographic Style}\\
    paperback, first edition, 1993
\bibitem[Eijkhout2008]{Eijkhout2008}
    \textbf{Victor Eijkhout}\\
    \textit{\TeX{} by Topic, a \TeX{}nician's Reference}\\
    document revision 1.2, may 2008\\
    \url{http://www.eijkhout.net/texbytopic/texbytopic.html}
\end{thebibliography}
%doc%
\subsection{French spacing}
%doc%
\paragraph{Why?} \cite[p 28, p 30]{Bringhurst1993}: `2.1.4 Use a single word space between sentences.'
%doc%
\paragraph{How?} \cite[p 185]{Eijkhout2008}:\\
\verb#\frenchspacing  %% Macro to switch off extra space after punctuation.# \\
%doc%
Note: This setting might be default for \myabk{KOMA} script.
%doc%
%doc%
\subsection{Text figures}

\ldots also called old style numbers. 
(German: Mediävalziffern\footnote{\url{https://secure.wikimedia.org/wikibooks/de/wiki/LaTeX-W\%C3\%B6rterbuch:\_Medi\%C3\%A4valziffern}})
\paragraph{Why?} \cite[p 44f]{Bringhurst1993}: 
\begin{quote}
`3.2.1 If the font includes both text figures and titling figures, use
 titling figures only with full caps, and text figures in all other
 circumstances.'
\end{quote}

\paragraph{How?} 
Quoted from Wikibooks\footnote{\url{https://secure.wikimedia.org/wikibooks/en/wiki/LaTeX/Formatting\#Text\_figures\_.28.22old\_style.22\_numerals.29}}:
\begin{quote}
Some fonts do not have text figures built in; the textcomp package attempts to
remedy this by effectively generating text figures from the currently-selected
font. Put \verb#\usepackage{textcomp}# in your preamble. textcomp also allows you to
use decimal points, properly formatted dollar signs, etc. within
\verb#\oldstylenums{}#.
\end{quote}
\ldots but proposed \LaTeX{} method does not work out well. Instead use:\\
\verb#\usepackage{hfoldsty}#  (enables text figures using additional font) or \\
\verb#\usepackage[sc,osf]{mathpazo}# (switches to Palatino font with small caps and old style figures enabled).
%doc%

\subsection{Abbrevations using \textsc{small caps}}

\paragraph{Why?} \cite[p 45f]{Bringhurst1993}: `3.2.2 For abbrevations and
acronyms in the midst of normal text, use spaced small caps.'

\paragraph{How?} Using the predefined macro \verb#\myabk{}# for things like
\myabk{UNO} or \myabk{UNESCO} using \verb#\myabk{UNO}# or \verb#\myabk{UNESCO}#.


\subsection{Colorized headings and links}

This document template is able to generate an output that uses colorized
headings, captions, page numbers, and links. The color named `DispositionColor'
used in this document is defined near the definition of package \texttt{color}
in the preamble (see section~\ref{subsec:miscpackages}). The changes required
for headings, page numbers, and captions are defined here.

Settings for colored links are handled by the definitions of the
\texttt{hyperref} package (see section~\ref{sec:pdf}).

%doc%
\section{\texttt{pdf\_settings.tex} --- Settings related to PDF output}
\label{sec:pdf}

The file \verb#preamble/pdf_settings.tex# basically contains the definitions for
the \href{http://tug.org/applications/hyperref/}{\texttt{hyperref} package}
including the
\href{http://www.ctan.org/tex-archive/macros/latex/required/graphics/}{\texttt{graphicx}
package}. Since these settings should be the last things of any \LaTeX{}
preamble, they got their own \TeX{} file which is included in \texttt{main.tex}.

\paragraph{What should I do with this file?} The settings in this file are
important for \myabk{PDF} output and including graphics. Do not exclude the
related \texttt{input} command in \texttt{main.tex}. But you might want to
modify some settings after you read the
\href{http://tug.org/applications/hyperref/}{documentation of the \texttt{hyperref} package}.

# in \texttt{main.tex}.

\subsection{What about modifying the template?}

This template provides an easy to start \LaTeX{} document template with sound
default settings. You can modify each setting any time. It is recommended that
you are familiar with the documentation of the command whose settings you want
to modify.

The following sections describe the settings and commands of this template and
gives a short overview of its features.
%doc%
\section{\texttt{preamble.tex} --- Main preamble file}
%doc%
In file \verb#preamble/preamble.tex# you will find the basic
definitions related to your document. This template uses the \myabk{KOMA} script
extension package of \LaTeX{}.

There are comments added to the \verb#\documentclass{}# definitions. Please
refer to the great documentation of \myabk{KOMA}\footnote{\texttt{scrguide.pdf} for
German users} for further details.

\paragraph{What should I do with this file?} For standard purposes you might
use the default values it provides. You must not remove its \texttt{include} command
in \texttt{main.tex} since it contains important definitions. This file contains
settings which are documented well an can be modified according to your needs.
It is recommended that you fully understand each setting you modify in order to
get a good document result.


\subsection{UTF8 as input charset}

You are able and should use \myabk{UTF8} character settings for writing these \TeX{}-files.


\subsection{Language settings}

The default setting of the language is English. Please change settings for
additional or alternative languages used.


\subsection{Headers and footers}

Since this template is based on \myabk{KOMA} script it uses its great \texttt{scrpage2}
package for defining header and footer information. Please refer to the \myabk{KOMA}
script documentation how to use this package.


\subsection{Miscellaneous packages} \label{subsec:miscpackages}

There are several packages included by default. You might want to activate or
deactivate them according to your requirements:

\begin{enumerate}
\item[\texttt{\href{https://secure.wikimedia.org/wikibooks/en/wiki/LaTeX/Formatting\#Other\_symbols}{%%
pifont%%
}}] 
For additional special characters available by \verb#\ding{}#
\item[\texttt{\href{http://ctan.org/pkg/ifthen}{%%
ifthen%%
}}] 
For using if/then/else statements for example in macros
\item[\texttt{\href{http://www.ctan.org/tex-archive/fonts/eurosym}{%%
eurosym%%
}}] 
Using the character for Euro with \verb#\officialeuro{}#
\item[\texttt{\href{http://www.ctan.org/tex-archive/help/Catalogue/entries/xspace.html}{%%
xspace%%
}}] 
This package is required for intelligent spacing after commands
\item[\texttt{\href{https://secure.wikimedia.org/wikibooks/en/wiki/LaTeX/Colors}{%%
color%%
}}] 
This package defines basic colors
\end{enumerate}
%doc%
\section{\texttt{typographic\_settings.tex} --- Typographic finetuning}
%doc%
The settings of file \verb#preamble/typographic_settings.tex# contain
typographic finetuning related to things mentioned in literature.  The
settings in this file relates to personal taste and most of all typographic
experience. 

\paragraph{What should I do with this file?} You might as well skip the whole
file by excluding the \verb#%%%% Time-stamp: <2011-07-22 15:45:04 vk>
%%%% === Disclaimer: =======================================================
%% created by
%%
%%      Karl Voit
%%
%% using grml GNU/Linux, vim & LaTeX 2e
%%
%doc%
%doc% \section{\texttt{typographic\_settings.tex} --- Typographic finetuning}
%doc%
%doc% The settings of file \verb#preamble/typographic_settings.tex# contain
%doc% typographic finetuning related to things mentioned in literature.  The
%doc% settings in this file relates to personal taste and most of all typographic
%doc% experience. 
%doc% 
%doc% \paragraph{What should I do with this file?} You might as well skip the whole
%doc% file by excluding the \verb#%%%% Time-stamp: <2011-07-22 15:45:04 vk>
%%%% === Disclaimer: =======================================================
%% created by
%%
%%      Karl Voit
%%
%% using grml GNU/Linux, vim & LaTeX 2e
%%
%doc%
%doc% \section{\texttt{typographic\_settings.tex} --- Typographic finetuning}
%doc%
%doc% The settings of file \verb#preamble/typographic_settings.tex# contain
%doc% typographic finetuning related to things mentioned in literature.  The
%doc% settings in this file relates to personal taste and most of all typographic
%doc% experience. 
%doc% 
%doc% \paragraph{What should I do with this file?} You might as well skip the whole
%doc% file by excluding the \verb#%%%% Time-stamp: <2011-07-22 15:45:04 vk>
%%%% === Disclaimer: =======================================================
%% created by
%%
%%      Karl Voit
%%
%% using grml GNU/Linux, vim & LaTeX 2e
%%
%doc%
%doc% \section{\texttt{typographic\_settings.tex} --- Typographic finetuning}
%doc%
%doc% The settings of file \verb#preamble/typographic_settings.tex# contain
%doc% typographic finetuning related to things mentioned in literature.  The
%doc% settings in this file relates to personal taste and most of all typographic
%doc% experience. 
%doc% 
%doc% \paragraph{What should I do with this file?} You might as well skip the whole
%doc% file by excluding the \verb#\input{preamble/typographic_settings.tex}# command
%doc% in \texttt{main.tex}.  For standard usage it is recommended to stay with the
%doc% default settings.
%doc% 
%doc% 
%doc% \subsection{References related to typographic settings}
%doc% 
%doc% \begin{thebibliography}{9}
%doc% \bibitem[Bringhurst1993]{Bringhurst1993}
%doc%     \textbf{Robert Bringhurst}\\
%doc%     \textit{The Elements of Typographic Style}\\
%doc%     paperback, first edition, 1993
%doc% \bibitem[Eijkhout2008]{Eijkhout2008}
%doc%     \textbf{Victor Eijkhout}\\
%doc%     \textit{\TeX{} by Topic, a \TeX{}nician's Reference}\\
%doc%     document revision 1.2, may 2008\\
%doc%     \url{http://www.eijkhout.net/texbytopic/texbytopic.html}
%doc% \end{thebibliography}
%% ========================================================================


%doc%
%doc% \subsection{French spacing}
%doc%
%doc% \paragraph{Why?} \cite[p 28, p 30]{Bringhurst1993}: `2.1.4 Use a single word space between sentences.'
%doc%
%doc% \paragraph{How?} \cite[p 185]{Eijkhout2008}:\\
%doc% \verb#\frenchspacing  %% Macro to switch off extra space after punctuation.# \\
\frenchspacing  %% Macro to switch off extra space after punctuation.
%doc%
%doc% Note: This setting might be default for \myabk{KOMA} script.
%doc%


%doc%
%doc% \subsection{Text figures}
%doc% 
%doc% \ldots also called old style numbers. 
%doc% (German: Mediävalziffern\footnote{\url{https://secure.wikimedia.org/wikibooks/de/wiki/LaTeX-W\%C3\%B6rterbuch:\_Medi\%C3\%A4valziffern}})
%doc% \paragraph{Why?} \cite[p 44f]{Bringhurst1993}: 
%doc% \begin{quote}
%doc% `3.2.1 If the font includes both text figures and titling figures, use
%doc%  titling figures only with full caps, and text figures in all other
%doc%  circumstances.'
%doc% \end{quote}
%doc% 
%doc% \paragraph{How?} 
%doc% Quoted from Wikibooks\footnote{\url{https://secure.wikimedia.org/wikibooks/en/wiki/LaTeX/Formatting\#Text\_figures\_.28.22old\_style.22\_numerals.29}}:
%doc% \begin{quote}
%doc% Some fonts do not have text figures built in; the textcomp package attempts to
%doc% remedy this by effectively generating text figures from the currently-selected
%doc% font. Put \verb#\usepackage{textcomp}# in your preamble. textcomp also allows you to
%doc% use decimal points, properly formatted dollar signs, etc. within
%doc% \verb#\oldstylenums{}#.
%doc% \end{quote}
%doc% \ldots but proposed \LaTeX{} method does not work out well. Instead use:\\
%doc% \verb#\usepackage{hfoldsty}#  (enables text figures using additional font) or \\
%doc% \verb#\usepackage[sc,osf]{mathpazo}# (switches to Palatino font with small caps and old style figures enabled).
%doc%
%\usepackage{hfoldsty}  %% enables text figures using additional font
%% ... OR use ...
\usepackage[sc,osf]{mathpazo} %% switches to Palatino with small caps and old style figures


%doc% 
%doc% \subsection{Abbrevations using \textsc{small caps}}
%doc% 
%doc% \paragraph{Why?} \cite[p 45f]{Bringhurst1993}: `3.2.2 For abbrevations and
%doc% acronyms in the midst of normal text, use spaced small caps.'
%doc% 
%doc% \paragraph{How?} Using the predefined macro \verb#\myabk{}# for things like
%doc% \myabk{UNO} or \myabk{UNESCO} using \verb#\myabk{UNO}# or \verb#\myabk{UNESCO}#.
%doc% 
\newcommand{\myabk}[1]{%%  abbrevations using small caps
\textsc{\lowercase{#1}}%%
}


%doc% 
%doc% \subsection{Colorized headings and links}
%doc% 
%doc% This document template is able to generate an output that uses colorized
%doc% headings, captions, page numbers, and links. The color named `DispositionColor'
%doc% used in this document is defined near the definition of package \texttt{color}
%doc% in the preamble (see section~\ref{subsec:miscpackages}). The changes required
%doc% for headings, page numbers, and captions are defined here.
%doc% 
%doc% Settings for colored links are handled by the definitions of the
%doc% \texttt{hyperref} package (see section~\ref{sec:pdf}).
%doc% 
\setheadsepline{.4pt}[\color{DispositionColor}]
\renewcommand{\headfont}{\normalfont\sffamily\color{DispositionColor}}
\renewcommand{\pnumfont}{\normalfont\sffamily\color{DispositionColor}}
\addtokomafont{disposition}{\color{DispositionColor}}
\addtokomafont{caption}{\color{DispositionColor}\footnotesize}
\addtokomafont{captionlabel}{\color{DispositionColor}}

%%%% END
%%% Local Variables:
%%% TeX-master: "../main"
%%% End:
%% vim:foldmethod=expr
%% vim:fde=getline(v\:lnum)=~'^%%%%'?0\:getline(v\:lnum)=~'^%doc.*\ .\\%(sub\\)\\?section{.\\+'?'>1'\:'1':
# command
%doc% in \texttt{main.tex}.  For standard usage it is recommended to stay with the
%doc% default settings.
%doc% 
%doc% 
%doc% \subsection{References related to typographic settings}
%doc% 
%doc% \begin{thebibliography}{9}
%doc% \bibitem[Bringhurst1993]{Bringhurst1993}
%doc%     \textbf{Robert Bringhurst}\\
%doc%     \textit{The Elements of Typographic Style}\\
%doc%     paperback, first edition, 1993
%doc% \bibitem[Eijkhout2008]{Eijkhout2008}
%doc%     \textbf{Victor Eijkhout}\\
%doc%     \textit{\TeX{} by Topic, a \TeX{}nician's Reference}\\
%doc%     document revision 1.2, may 2008\\
%doc%     \url{http://www.eijkhout.net/texbytopic/texbytopic.html}
%doc% \end{thebibliography}
%% ========================================================================


%doc%
%doc% \subsection{French spacing}
%doc%
%doc% \paragraph{Why?} \cite[p 28, p 30]{Bringhurst1993}: `2.1.4 Use a single word space between sentences.'
%doc%
%doc% \paragraph{How?} \cite[p 185]{Eijkhout2008}:\\
%doc% \verb#\frenchspacing  %% Macro to switch off extra space after punctuation.# \\
\frenchspacing  %% Macro to switch off extra space after punctuation.
%doc%
%doc% Note: This setting might be default for \myabk{KOMA} script.
%doc%


%doc%
%doc% \subsection{Text figures}
%doc% 
%doc% \ldots also called old style numbers. 
%doc% (German: Mediävalziffern\footnote{\url{https://secure.wikimedia.org/wikibooks/de/wiki/LaTeX-W\%C3\%B6rterbuch:\_Medi\%C3\%A4valziffern}})
%doc% \paragraph{Why?} \cite[p 44f]{Bringhurst1993}: 
%doc% \begin{quote}
%doc% `3.2.1 If the font includes both text figures and titling figures, use
%doc%  titling figures only with full caps, and text figures in all other
%doc%  circumstances.'
%doc% \end{quote}
%doc% 
%doc% \paragraph{How?} 
%doc% Quoted from Wikibooks\footnote{\url{https://secure.wikimedia.org/wikibooks/en/wiki/LaTeX/Formatting\#Text\_figures\_.28.22old\_style.22\_numerals.29}}:
%doc% \begin{quote}
%doc% Some fonts do not have text figures built in; the textcomp package attempts to
%doc% remedy this by effectively generating text figures from the currently-selected
%doc% font. Put \verb#\usepackage{textcomp}# in your preamble. textcomp also allows you to
%doc% use decimal points, properly formatted dollar signs, etc. within
%doc% \verb#\oldstylenums{}#.
%doc% \end{quote}
%doc% \ldots but proposed \LaTeX{} method does not work out well. Instead use:\\
%doc% \verb#\usepackage{hfoldsty}#  (enables text figures using additional font) or \\
%doc% \verb#\usepackage[sc,osf]{mathpazo}# (switches to Palatino font with small caps and old style figures enabled).
%doc%
%\usepackage{hfoldsty}  %% enables text figures using additional font
%% ... OR use ...
\usepackage[sc,osf]{mathpazo} %% switches to Palatino with small caps and old style figures


%doc% 
%doc% \subsection{Abbrevations using \textsc{small caps}}
%doc% 
%doc% \paragraph{Why?} \cite[p 45f]{Bringhurst1993}: `3.2.2 For abbrevations and
%doc% acronyms in the midst of normal text, use spaced small caps.'
%doc% 
%doc% \paragraph{How?} Using the predefined macro \verb#\myabk{}# for things like
%doc% \myabk{UNO} or \myabk{UNESCO} using \verb#\myabk{UNO}# or \verb#\myabk{UNESCO}#.
%doc% 
\newcommand{\myabk}[1]{%%  abbrevations using small caps
\textsc{\lowercase{#1}}%%
}


%doc% 
%doc% \subsection{Colorized headings and links}
%doc% 
%doc% This document template is able to generate an output that uses colorized
%doc% headings, captions, page numbers, and links. The color named `DispositionColor'
%doc% used in this document is defined near the definition of package \texttt{color}
%doc% in the preamble (see section~\ref{subsec:miscpackages}). The changes required
%doc% for headings, page numbers, and captions are defined here.
%doc% 
%doc% Settings for colored links are handled by the definitions of the
%doc% \texttt{hyperref} package (see section~\ref{sec:pdf}).
%doc% 
\setheadsepline{.4pt}[\color{DispositionColor}]
\renewcommand{\headfont}{\normalfont\sffamily\color{DispositionColor}}
\renewcommand{\pnumfont}{\normalfont\sffamily\color{DispositionColor}}
\addtokomafont{disposition}{\color{DispositionColor}}
\addtokomafont{caption}{\color{DispositionColor}\footnotesize}
\addtokomafont{captionlabel}{\color{DispositionColor}}

%%%% END
%%% Local Variables:
%%% TeX-master: "../main"
%%% End:
%% vim:foldmethod=expr
%% vim:fde=getline(v\:lnum)=~'^%%%%'?0\:getline(v\:lnum)=~'^%doc.*\ .\\%(sub\\)\\?section{.\\+'?'>1'\:'1':
# command
%doc% in \texttt{main.tex}.  For standard usage it is recommended to stay with the
%doc% default settings.
%doc% 
%doc% 
%doc% \subsection{References related to typographic settings}
%doc% 
%doc% \begin{thebibliography}{9}
%doc% \bibitem[Bringhurst1993]{Bringhurst1993}
%doc%     \textbf{Robert Bringhurst}\\
%doc%     \textit{The Elements of Typographic Style}\\
%doc%     paperback, first edition, 1993
%doc% \bibitem[Eijkhout2008]{Eijkhout2008}
%doc%     \textbf{Victor Eijkhout}\\
%doc%     \textit{\TeX{} by Topic, a \TeX{}nician's Reference}\\
%doc%     document revision 1.2, may 2008\\
%doc%     \url{http://www.eijkhout.net/texbytopic/texbytopic.html}
%doc% \end{thebibliography}
%% ========================================================================


%doc%
%doc% \subsection{French spacing}
%doc%
%doc% \paragraph{Why?} \cite[p 28, p 30]{Bringhurst1993}: `2.1.4 Use a single word space between sentences.'
%doc%
%doc% \paragraph{How?} \cite[p 185]{Eijkhout2008}:\\
%doc% \verb#\frenchspacing  %% Macro to switch off extra space after punctuation.# \\
\frenchspacing  %% Macro to switch off extra space after punctuation.
%doc%
%doc% Note: This setting might be default for \myabk{KOMA} script.
%doc%


%doc%
%doc% \subsection{Text figures}
%doc% 
%doc% \ldots also called old style numbers. 
%doc% (German: Mediävalziffern\footnote{\url{https://secure.wikimedia.org/wikibooks/de/wiki/LaTeX-W\%C3\%B6rterbuch:\_Medi\%C3\%A4valziffern}})
%doc% \paragraph{Why?} \cite[p 44f]{Bringhurst1993}: 
%doc% \begin{quote}
%doc% `3.2.1 If the font includes both text figures and titling figures, use
%doc%  titling figures only with full caps, and text figures in all other
%doc%  circumstances.'
%doc% \end{quote}
%doc% 
%doc% \paragraph{How?} 
%doc% Quoted from Wikibooks\footnote{\url{https://secure.wikimedia.org/wikibooks/en/wiki/LaTeX/Formatting\#Text\_figures\_.28.22old\_style.22\_numerals.29}}:
%doc% \begin{quote}
%doc% Some fonts do not have text figures built in; the textcomp package attempts to
%doc% remedy this by effectively generating text figures from the currently-selected
%doc% font. Put \verb#\usepackage{textcomp}# in your preamble. textcomp also allows you to
%doc% use decimal points, properly formatted dollar signs, etc. within
%doc% \verb#\oldstylenums{}#.
%doc% \end{quote}
%doc% \ldots but proposed \LaTeX{} method does not work out well. Instead use:\\
%doc% \verb#\usepackage{hfoldsty}#  (enables text figures using additional font) or \\
%doc% \verb#\usepackage[sc,osf]{mathpazo}# (switches to Palatino font with small caps and old style figures enabled).
%doc%
%\usepackage{hfoldsty}  %% enables text figures using additional font
%% ... OR use ...
\usepackage[sc,osf]{mathpazo} %% switches to Palatino with small caps and old style figures


%doc% 
%doc% \subsection{Abbrevations using \textsc{small caps}}
%doc% 
%doc% \paragraph{Why?} \cite[p 45f]{Bringhurst1993}: `3.2.2 For abbrevations and
%doc% acronyms in the midst of normal text, use spaced small caps.'
%doc% 
%doc% \paragraph{How?} Using the predefined macro \verb#\myabk{}# for things like
%doc% \myabk{UNO} or \myabk{UNESCO} using \verb#\myabk{UNO}# or \verb#\myabk{UNESCO}#.
%doc% 
\newcommand{\myabk}[1]{%%  abbrevations using small caps
\textsc{\lowercase{#1}}%%
}


%doc% 
%doc% \subsection{Colorized headings and links}
%doc% 
%doc% This document template is able to generate an output that uses colorized
%doc% headings, captions, page numbers, and links. The color named `DispositionColor'
%doc% used in this document is defined near the definition of package \texttt{color}
%doc% in the preamble (see section~\ref{subsec:miscpackages}). The changes required
%doc% for headings, page numbers, and captions are defined here.
%doc% 
%doc% Settings for colored links are handled by the definitions of the
%doc% \texttt{hyperref} package (see section~\ref{sec:pdf}).
%doc% 
\setheadsepline{.4pt}[\color{DispositionColor}]
\renewcommand{\headfont}{\normalfont\sffamily\color{DispositionColor}}
\renewcommand{\pnumfont}{\normalfont\sffamily\color{DispositionColor}}
\addtokomafont{disposition}{\color{DispositionColor}}
\addtokomafont{caption}{\color{DispositionColor}\footnotesize}
\addtokomafont{captionlabel}{\color{DispositionColor}}

%%%% END
%%% Local Variables:
%%% TeX-master: "../main"
%%% End:
%% vim:foldmethod=expr
%% vim:fde=getline(v\:lnum)=~'^%%%%'?0\:getline(v\:lnum)=~'^%doc.*\ .\\%(sub\\)\\?section{.\\+'?'>1'\:'1':
# command
in \texttt{main.tex}.  For standard usage it is recommended to stay with the
default settings.


\subsection{References related to typographic settings}

\begin{thebibliography}{9}
\bibitem[Bringhurst1993]{Bringhurst1993}
    \textbf{Robert Bringhurst}\\
    \textit{The Elements of Typographic Style}\\
    paperback, first edition, 1993
\bibitem[Eijkhout2008]{Eijkhout2008}
    \textbf{Victor Eijkhout}\\
    \textit{\TeX{} by Topic, a \TeX{}nician's Reference}\\
    document revision 1.2, may 2008\\
    \url{http://www.eijkhout.net/texbytopic/texbytopic.html}
\end{thebibliography}
%doc%
\subsection{French spacing}
%doc%
\paragraph{Why?} \cite[p 28, p 30]{Bringhurst1993}: `2.1.4 Use a single word space between sentences.'
%doc%
\paragraph{How?} \cite[p 185]{Eijkhout2008}:\\
\verb#\frenchspacing  %% Macro to switch off extra space after punctuation.# \\
%doc%
Note: This setting might be default for \myabk{KOMA} script.
%doc%
%doc%
\subsection{Text figures}

\ldots also called old style numbers. 
(German: Mediävalziffern\footnote{\url{https://secure.wikimedia.org/wikibooks/de/wiki/LaTeX-W\%C3\%B6rterbuch:\_Medi\%C3\%A4valziffern}})
\paragraph{Why?} \cite[p 44f]{Bringhurst1993}: 
\begin{quote}
`3.2.1 If the font includes both text figures and titling figures, use
 titling figures only with full caps, and text figures in all other
 circumstances.'
\end{quote}

\paragraph{How?} 
Quoted from Wikibooks\footnote{\url{https://secure.wikimedia.org/wikibooks/en/wiki/LaTeX/Formatting\#Text\_figures\_.28.22old\_style.22\_numerals.29}}:
\begin{quote}
Some fonts do not have text figures built in; the textcomp package attempts to
remedy this by effectively generating text figures from the currently-selected
font. Put \verb#\usepackage{textcomp}# in your preamble. textcomp also allows you to
use decimal points, properly formatted dollar signs, etc. within
\verb#\oldstylenums{}#.
\end{quote}
\ldots but proposed \LaTeX{} method does not work out well. Instead use:\\
\verb#\usepackage{hfoldsty}#  (enables text figures using additional font) or \\
\verb#\usepackage[sc,osf]{mathpazo}# (switches to Palatino font with small caps and old style figures enabled).
%doc%

\subsection{Abbrevations using \textsc{small caps}}

\paragraph{Why?} \cite[p 45f]{Bringhurst1993}: `3.2.2 For abbrevations and
acronyms in the midst of normal text, use spaced small caps.'

\paragraph{How?} Using the predefined macro \verb#\myabk{}# for things like
\myabk{UNO} or \myabk{UNESCO} using \verb#\myabk{UNO}# or \verb#\myabk{UNESCO}#.


\subsection{Colorized headings and links}

This document template is able to generate an output that uses colorized
headings, captions, page numbers, and links. The color named `DispositionColor'
used in this document is defined near the definition of package \texttt{color}
in the preamble (see section~\ref{subsec:miscpackages}). The changes required
for headings, page numbers, and captions are defined here.

Settings for colored links are handled by the definitions of the
\texttt{hyperref} package (see section~\ref{sec:pdf}).

%doc%
\section{\texttt{pdf\_settings.tex} --- Settings related to PDF output}
\label{sec:pdf}

The file \verb#preamble/pdf_settings.tex# basically contains the definitions for
the \href{http://tug.org/applications/hyperref/}{\texttt{hyperref} package}
including the
\href{http://www.ctan.org/tex-archive/macros/latex/required/graphics/}{\texttt{graphicx}
package}. Since these settings should be the last things of any \LaTeX{}
preamble, they got their own \TeX{} file which is included in \texttt{main.tex}.

\paragraph{What should I do with this file?} The settings in this file are
important for \myabk{PDF} output and including graphics. Do not exclude the
related \texttt{input} command in \texttt{main.tex}. But you might want to
modify some settings after you read the
\href{http://tug.org/applications/hyperref/}{documentation of the \texttt{hyperref} package}.

