%% Time-stamp: <2013-04-24 10:16:48 vk>
%%%% === Disclaimer: =======================================================
%% created by
%%
%%      Karl Voit
%%
%% using GNU/Linux, GNU Emacs & LaTeX 2e
%%

%doc% %% overriding preamble/preamble.tex %%
%doc% \newcommand{\mylinespread}{1.0}  \newcommand{\mycolorlinks}{true}
%doc% \documentclass[12pt,paper=a4,parskip=half,DIV=calc,oneside,%%
%doc% headinclude,footinclude=false,open=right,bibliography=totoc]{scrartcl}
%doc% \usepackage[utf8]{inputenc}\usepackage[ngerman,american]{babel}\usepackage{scrpage2}
%doc% \usepackage{ifthen}\usepackage{eurosym}\usepackage{xspace}\usepackage[usenames,dvipsnames]{xcolor}
%doc% \usepackage[protrusion=true,factor=900]{microtype}
%doc% \usepackage{enumitem}
%doc% \usepackage[pdftex]{graphicx}
%doc% \usepackage{todonotes}
%doc% \usepackage{dingbat,bbding} %% special characters
%doc% \definecolor{DispositionColor}{RGB}{30,103,182}
%doc% 
%doc% \usepackage[backend=biber,style=authoryear,dashed=false,natbib=true,hyperref=true%%
%doc% ]{biblatex}
%doc% 
%doc% \addbibresource{references-biblatex.bib} %% remove, if using BibTeX instead of biblatex
%doc% 
%doc% %% overriding userdata %%
%doc% \newcommand{\myauthor}{Karl Voit}\newcommand{\mytitle}{LaTeX Template Documentation}
%doc% \newcommand{\mysubject}{A Comprehensive Guide to Use the
%doc% Template from https://github.com/novoid/LaTeX-KOMA-template}
%doc% \newcommand{\mykeywords}{LaTeX, pdflatex, template, documentation, biber, biblatex}
%doc% 
%doc% \newcommand{\myLaT}{\LaTeX{}@TUG\xspace}
%doc% 
%doc% %% for future use?
%doc% % \usepackage{filecontents}
%doc% % \begin{filecontents}{filename.example}
%doc% % 
%doc% % \end{filecontents}
%doc% 
%doc% 
%doc% %% using existing TeX files %%
%doc% %% Time-stamp: <2015-04-30 17:19:58 vk>
%%%% === Disclaimer: =======================================================
%% created by
%%
%%      Karl Voit
%%
%% using GNU/Linux, GNU Emacs & LaTeX 2e
%%

%doc%
%doc% \section{\texttt{mycommands.tex} --- various definitions}\myinteresting
%doc% \label{sec:mycommands}
%doc%
%doc% In file \verb#template/mycommands.tex# many useful commands are being
%doc% defined. 
%doc% 
%doc% \paragraph{What should I do with this file?} Please take a look at its 
%doc% content to get the most out of your document.
%doc% 

%doc% 
%doc% One of the best advantages of \LaTeX{} compared to \myacro{WYSIWYG} software products is
%doc% the possibility to define and use macros within text. This empowers the user to
%doc% a great extend.  Many things can be defined using \verb#\newcommand{}# and
%doc% automates repeating tasks. It is recommended to use macros not only for
%doc% repetitive tasks but also for separating form from content such as \myacro{CSS}
%doc% does for \myacro{XHTML}. Think of including graphics in your document: after
%doc% writing your book, you might want to change all captions to the upper side of
%doc% each figure. In this case you either have to modify all
%doc% \texttt{includegraphics} commands or you were clever enough to define something
%doc% like \verb#\myfig#\footnote{See below for a detailed description}. Using a
%doc% macro for including graphics enables you to modify the position caption on only
%doc% \emph{one} place: at the definition of the macro.
%doc% 
%doc% The following section describes some macros that came with this document template
%doc% from \myLaT and you are welcome to modify or extend them or to create
%doc% your own macros!
%doc% 

%doc% 
%doc% \subsection{\texttt{myfig} --- including graphics made easy}
%doc% 
%doc% The classic: you can easily add graphics to your document with \verb#\myfig#:
%doc% \begin{verbatim}
%doc%  \myfig{flower}%% filename w/o extension in the folder figures
%doc%        {width=0.7\textwidth}%% maximum width/height, aspect ratio will be kept
%doc%        {This flower was photographed at my home town in 2010}%% caption
%doc%        {Home town flower}%% optional (short) caption for list of figures
%doc%        {fig:flower}%% label
%doc% \end{verbatim}
%doc% 
%doc% There are many advantages of this command (compared to manual
%doc% \texttt{figure} environments and \texttt{includegraphics} commands:
%doc% \begin{itemize}
%doc% \item consistent style throughout the whole document
%doc% \item easy to change; for example move caption on top
%doc% \item much less characters to type (faster, error prone)
%doc% \item less visual clutter in the \TeX{}-files
%doc% \end{itemize}
%doc% 
%doc% 
\newcommand{\myfig}[5]{
%% example:
% \myfig{}%% filename in figures folder
%       {width=0.5\textwidth,height=0.5\textheight}%% maximum width/height, aspect ratio will be kept
%       {}%% caption
%       {}%% optional (short) caption for list of figures
%       {}%% label
\begin{figure}%[htp]
  \centering
  \includegraphics[keepaspectratio,#2]{figures/#1}
  \caption[#4]{#3}
  \label{#5} %% NOTE: always label *after* caption!
\end{figure}
}


%doc% 
%doc% \subsection{\texttt{myclone} --- repeat things!}
%doc% 
%doc% Using \verb#\myclone[42]{foobar}# results the text \enquote{foobar} printed 42 times.
%doc% But you can not only repeat text output with \texttt{myclone}. 
%doc%
%doc% Default argument
%doc% for the optional parameter \enquote{number of times} (like \enquote{42} in the example above) 
%doc% is set to two.
%doc% 
%% \myclone[x]{text}
\newcounter{myclonecnt}
\newcommand{\myclone}[2][2]{%
  \setcounter{myclonecnt}{#1}%
  \whiledo{\value{myclonecnt}>0}{#2\addtocounter{myclonecnt}{-1}}%
}

%old% %d oc% 
%old% %d oc% \subsection{\texttt{fixxme} --- sidemark something as unfinished}
%old% %d oc% 
%old% %d oc% You know it: something has to be fixed and you can not do it right
%old% %d oc% now. In order to \texttt{not} forget about it, you might want to add a
%old% %d oc% note like \verb+\fixxme{check again}+ which inserts a note on the page
%old% %d oc% margin such as this\fixxme{check again} example.
%old% %d oc%
%old% \newcommand{\fixxme}[1]{%%
%old% \textcolor{red}{FIXXME}\marginpar{\textcolor{red}{#1}}%%
%old% }


%%%% End 
%%% Local Variables:
%%% mode: latex
%%% mode: auto-fill
%%% mode: flyspell
%%% eval: (ispell-change-dictionary "en_US")
%%% TeX-master: "../main"
%%% End:
%% vim:foldmethod=expr
%% vim:fde=getline(v\:lnum)=~'^%%%%'?0\:getline(v\:lnum)=~'^%doc.*\ .\\%(sub\\)\\?section{.\\+'?'>1'\:'1':

%doc% %%%% Time-stamp: <2013-02-26 12:18:29 vk>
%%%% === Disclaimer: =======================================================
%% created by
%%
%%      Karl Voit
%%
%% using GNU/Linux, GNU Emacs & LaTeX 2e
%%
%doc%
%doc% \section{\texttt{typographic\_settings.tex} --- Typographic finetuning}
%doc%
%doc% The settings of file \verb#template/typographic_settings.tex# contain
%doc% typographic finetuning related to things mentioned in literature.  The
%doc% settings in this file relates to personal taste and most of all: 
%doc% \emph{typographic experience}. 
%doc% 
%doc% \paragraph{What should I do with this file?} You might as well skip the whole
%doc% file by excluding the \verb#%%%% Time-stamp: <2013-02-26 12:18:29 vk>
%%%% === Disclaimer: =======================================================
%% created by
%%
%%      Karl Voit
%%
%% using GNU/Linux, GNU Emacs & LaTeX 2e
%%
%doc%
%doc% \section{\texttt{typographic\_settings.tex} --- Typographic finetuning}
%doc%
%doc% The settings of file \verb#template/typographic_settings.tex# contain
%doc% typographic finetuning related to things mentioned in literature.  The
%doc% settings in this file relates to personal taste and most of all: 
%doc% \emph{typographic experience}. 
%doc% 
%doc% \paragraph{What should I do with this file?} You might as well skip the whole
%doc% file by excluding the \verb#%%%% Time-stamp: <2013-02-26 12:18:29 vk>
%%%% === Disclaimer: =======================================================
%% created by
%%
%%      Karl Voit
%%
%% using GNU/Linux, GNU Emacs & LaTeX 2e
%%
%doc%
%doc% \section{\texttt{typographic\_settings.tex} --- Typographic finetuning}
%doc%
%doc% The settings of file \verb#template/typographic_settings.tex# contain
%doc% typographic finetuning related to things mentioned in literature.  The
%doc% settings in this file relates to personal taste and most of all: 
%doc% \emph{typographic experience}. 
%doc% 
%doc% \paragraph{What should I do with this file?} You might as well skip the whole
%doc% file by excluding the \verb#\input{template/typographic_settings.tex}# command
%doc% in \texttt{main.tex}.  For standard usage it is recommended to stay with the
%doc% default settings.
%doc% 
%doc% 
%% ========================================================================

%doc%
%doc% Some basic microtypographic settings are provided by the
%doc% \texttt{microtype}
%doc% package\footnote{\url{http://ctan.org/pkg/microtype}}. This template
%doc% uses the rather conservative package parameters: \texttt{protrusion=true,factor=900}.
\usepackage[protrusion=true,factor=900]{microtype}

%doc%
%doc% \subsection{French spacing}
%doc%
%doc% \paragraph{Why?} see~\textcite[p.\,28, p.\,30]{Bringhurst1993}: `2.1.4 Use a single word space between sentences.'
%doc%
%doc% \paragraph{How?} see~\textcite[p.\,185]{Eijkhout2008}:\\
%doc% \verb#\frenchspacing  %% Macro to switch off extra space after punctuation.# \\
\frenchspacing  %% Macro to switch off extra space after punctuation.
%doc%
%doc% Note: This setting might be default for \myacro{KOMA} script.
%doc%


%doc%
%doc% \subsection{Font}
%doc% 
%doc% This template is using the Palatino font (package \texttt{mathpazo}) which results
%doc% in a legible document and matching mathematical fonts for printout.
%doc% 
%doc% It is highly recommended that you either stick to the Palatino font or use the
%doc% \LaTeX{} default fonts (by removing the package \texttt{mathpazo}).
%doc% 
%doc% Chosing different fonts is not
%doc% an easy task. Please leave this to people with good knowledge on this subject.
%doc% 
%doc% One valid reason to change the default fonts is when your document is mainly
%doc% read on a computer screen. In this case it is recommended to switch to a font
%doc% \textsf{which is sans-serif like this}. This template contains several alternative
%doc% font packages which can be activated in this file.
%doc% 

% for changing the default font, please go to the next subsection!

%doc%
%doc% \subsection{Text figures}
%doc% 
%doc% \ldots also called old style numbers such as 0123456789. 
%doc% (German: \enquote{Mediäval\-ziffern\footnote{\url{https://secure.wikimedia.org/wikibooks/de/wiki/LaTeX-W\%C3\%B6rterbuch:\_Medi\%C3\%A4valziffern}}})
%doc% \paragraph{Why?} see~\textcite[p.\,44f]{Bringhurst1993}: 
%doc% \begin{quote}
%doc% `3.2.1 If the font includes both text figures and titling figures, use
%doc%  titling figures only with full caps, and text figures in all other
%doc%  circumstances.'
%doc% \end{quote}
%doc% 
%doc% \paragraph{How?} 
%doc% Quoted from Wikibooks\footnote{\url{https://secure.wikimedia.org/wikibooks/en/wiki/LaTeX/Formatting\#Text\_figures\_.28.22old\_style.22\_numerals.29}}:
%doc% \begin{quote}
%doc% Some fonts do not have text figures built in; the textcomp package attempts to
%doc% remedy this by effectively generating text figures from the currently-selected
%doc% font. Put \verb#\usepackage{textcomp}# in your preamble. textcomp also allows you to
%doc% use decimal points, properly formatted dollar signs, etc. within
%doc% \verb#\oldstylenums{}#.
%doc% \end{quote}
%doc% \ldots but proposed \LaTeX{} method does not work out well. Instead use:\\
%doc% \verb#\usepackage{hfoldsty}#  (enables text figures using additional font) or \\
%doc% \verb#\usepackage[sc,osf]{mathpazo}# (switches to Palatino font with small caps and old style figures enabled).
%doc%
%\usepackage{hfoldsty}  %% enables text figures using additional font
%% ... OR use ...
\usepackage[sc,osf]{mathpazo} %% switches to Palatino with small caps and old style figures

%% Font selection from:
%%     http://www.matthiaspospiech.de/latex/vorlagen/allgemein/preambel/fonts/
%% use following lines *instead* of the mathpazo package above:
%% ===== Serif =========================================================
%% for Computer Modern (LaTeX default font), simply remove the mathpazo above
%\usepackage{charter}\linespread{1.05} %% Charter
%\usepackage{bookman}                  %% Bookman (laedt Avant Garde !!)
%\usepackage{newcent}                  %% New Century Schoolbook (laedt Avant Garde !!)
%% ===== Sans Serif ====================================================
%\renewcommand{\familydefault}{\sfdefault}  %% this one in *combination* with the default mathpazo package
%\usepackage{cmbright}                  %% CM-Bright (eigntlich eine Familie)
%\usepackage{tpslifonts}                %% tpslifonts % Font for Slides


%doc% 
%doc% \subsection{\texttt{myacro} --- Abbrevations using \textsc{small caps}}\myinteresting
%doc% \label{sec:myacro}
%doc% 
%doc% \paragraph{Why?} see~\textcite[p.\,45f]{Bringhurst1993}: `3.2.2 For abbrevations and
%doc% acronyms in the midst of normal text, use spaced small caps.'
%doc% 
%doc% \paragraph{How?} Using the predefined macro \verb#\myacro{}# for things like
%doc% \myacro{UNO} or \myacro{UNESCO} using \verb#\myacro{UNO}# or \verb#\myacro{UNESCO}#.
%doc% 
\DeclareRobustCommand{\myacro}[1]{\textsc{\lowercase{#1}}} %%  abbrevations using small caps


%doc% 
%doc% \subsection{Colorized headings and links}
%doc% 
%doc% This document template is able to generate an output that uses colorized
%doc% headings, captions, page numbers, and links. The color named `DispositionColor'
%doc% used in this document is defined near the definition of package \texttt{color}
%doc% in the preamble (see section~\ref{subsec:miscpackages}). The changes required
%doc% for headings, page numbers, and captions are defined here.
%doc% 
%doc% Settings for colored links are handled by the definitions of the
%doc% \texttt{hyperref} package (see section~\ref{sec:pdf}).
%doc% 
\setheadsepline{.4pt}[\color{DispositionColor}]
\renewcommand{\headfont}{\normalfont\sffamily\color{DispositionColor}}
\renewcommand{\pnumfont}{\normalfont\sffamily\color{DispositionColor}}
\addtokomafont{disposition}{\color{DispositionColor}}
\addtokomafont{caption}{\color{DispositionColor}\footnotesize}
\addtokomafont{captionlabel}{\color{DispositionColor}}

%doc% 
%doc% \subsection{No figures or tables below footnotes}
%doc% 
%doc% \LaTeX{} places floating environments below footnotes if \texttt{b}
%doc% (bottom) is used as (default) placement algorithm. This is certainly
%doc% not appealing for most people and is deactivated in this template by
%doc% using the package \texttt{footmisc} with its option \texttt{bottom}.
%doc% 
%% see also: http://www.komascript.de/node/858 (German description)
\usepackage[bottom]{footmisc}

%doc% 
%doc% \subsection{Spacings of list environments}
%doc% 
%doc% By default, \LaTeX{} is using vertical spaces between items of enumerate, 
%doc% itemize and description environments. This is fine for multi-line items.
%doc% Many times, the user does just write single-line items where the larger
%doc% vertical space is inappropriate. The \href{http://ctan.org/pkg/enumitem}{enumitem}
%doc% package provides replacements for the pre-defined list environments and
%doc% offers many options to modify their appearances.
%doc% This template is using the package option for \texttt{noitemsep} which
%doc% mimimizes the vertical space between list items.
%doc% 
\usepackage{enumitem}
\setlist{noitemsep}   %% kills the space between items

%doc% 
%doc% \subsection{\texttt{csquotes} --- Correct quotation marks}\myinteresting
%doc% \label{sub:csquotes}
%doc% 
%doc% \emph{Never} use quotation marks found on your keyboard.
%doc% They end up in strange characters or false looking quotation marks.
%doc% 
%doc% In \LaTeX{} you are able to use typographically correct quotation marks. The package 
%doc% \href{http://www.ctan.org/pkg/csquotes}{\texttt{csquotes}} offers you with 
%doc% \verb#\enquote{foobar}# a command to get correct quotation marks around \enquote{foobar}.
%doc% Please do check the package options in order to modify
%doc% its settings according to the language used\footnote{most of the time in 
%doc% combination with the language set in the options of the \texttt{babel} package}.
%doc% 
%doc% \href{http://www.ctan.org/pkg/csquotes}{\texttt{csquotes}} is also recommended 
%doc% by \texttt{biblatex} (see Section~\ref{sec:references}). 
\usepackage[babel=true,strict=true,english=american,german=guillemets]{csquotes}

%doc% 
%doc% \subsection{Line spread}
%doc% 
%doc% If you have to enlarge the distance between two lines of text, you can
%doc% increase it using the \texttt{\mylinespread} command in \texttt{main.tex}. By default, it is
%doc% deactivated (set to 100~percent). Modify only with caution since it influences the
%doc% page layout and could lead to ugly looking documents.
\linespread{\mylinespread}

%doc% 
%doc% \subsection{Optional: Lines above and below the chapter head}
%doc% 
%doc% This is not quite something typographic but rather a matter of taste.
%doc% \myacro{KOMA} Script offers \href{http://www.komascript.de/node/24}{a method to
%doc% add lines above and below chapter head} which is disabled by
%doc% default. If you want to enable this feature, remove corresponding
%doc% comment characters from the settings.
%doc% 
%% Source: http://www.komascript.de/node/24
%disabled% %% 1st get a new command
%disabled% \newcommand*{\ORIGchapterheadstartvskip}{}%
%disabled% %% 2nd save the original definition to the new command
%disabled% \let\ORIGchapterheadstartvskip=\chapterheadstartvskip
%disabled% %% 3rd redefine the command using the saved original command
%disabled% \renewcommand*{\chapterheadstartvskip}{%
%disabled%   \ORIGchapterheadstartvskip
%disabled%   {%
%disabled%     \setlength{\parskip}{0pt}%
%disabled%     \noindent\color{DispositionColor}\rule[.3\baselineskip]{\linewidth}{1pt}\par
%disabled%   }%
%disabled% }
%disabled% %% see above
%disabled% \newcommand*{\ORIGchapterheadendvskip}{}%
%disabled% \let\ORIGchapterheadendvskip=\chapterheadendvskip
%disabled% \renewcommand*{\chapterheadendvskip}{%
%disabled%   {%
%disabled%     \setlength{\parskip}{0pt}%
%disabled%     \noindent\color{DispositionColor}\rule[.3\baselineskip]{\linewidth}{1pt}\par
%disabled%   }%
%disabled%   \ORIGchapterheadendvskip
%disabled% }

%doc% 
%doc% \subsection{Optional: Chapter thumbs}
%doc% 
%doc% This is not quite something typographic but rather a matter of taste.
%doc% \myacro{KOMA} Script offers \href{http://www.komascript.de/chapterthumbs-example}{a method to
%doc% add chapter thumbs} (in combination with the package \texttt{scrpage2}) which is disabled by
%doc% default. If you want to enable this feature, remove corresponding
%doc% comment characters from the settings.
%doc% 
%disabled% % Safty first
%disabled% \@ifundefined{chapter}{\let\chapter\undefined
%disabled%   \chapter must be defined to use chapter thumbs!}{%
%disabled%  
%disabled% % Two new commands for the width and height of the boxes with the
%disabled% % chapter number at the thumbs (use of commands instead of lengths
%disabled% % for sparing registers)
%disabled% \newcommand*{\chapterthumbwidth}{2em}
%disabled% \newcommand*{\chapterthumbheight}{1em}
%disabled%  
%disabled% % Two new commands for the colors of the box background and the
%disabled% % chapter numbers of the thumbs
%disabled% \newcommand*{\chapterthumbboxcolor}{black}
%disabled% \newcommand*{\chapterthumbtextcolor}{white}
%disabled%  
%disabled% % New command to set a chapter thumb. I'm using a group at this
%disabled% % command, because I'm changing the temporary dimension \@tempdima
%disabled% \newcommand*{\putchapterthumb}{%
%disabled%   \begingroup
%disabled%     \Large
%disabled%     % calculate the horizontal possition of the right paper border
%disabled%     % (I ignore \hoffset, because I interprete \hoffset moves the page
%disabled%     % at the paper e.g. if you are using cropmarks)
%disabled%     \setlength{\@tempdima}{\@oddheadshift}% (internal from scrpage2)
%disabled%     \setlength{\@tempdima}{-\@tempdima}%
%disabled%     \addtolength{\@tempdima}{\paperwidth}%
%disabled%     \addtolength{\@tempdima}{-\oddsidemargin}%
%disabled%     \addtolength{\@tempdima}{-1in}%
%disabled%     % putting the thumbs should not change the horizontal
%disabled%     % possition
%disabled%     \rlap{%
%disabled%       % move to the calculated horizontal possition
%disabled%       \hspace*{\@tempdima}%
%disabled%       % putting the thumbs should not change the vertical
%disabled%       % possition
%disabled%       \vbox to 0pt{%
%disabled%         % calculate the vertical possition of the thumbs (I ignore
%disabled%         % \voffset for the same reasons told above)
%disabled%         \setlength{\@tempdima}{\chapterthumbwidth}%
%disabled%         \multiply\@tempdima by\value{chapter}%
%disabled%         \addtolength{\@tempdima}{-\chapterthumbwidth}%
%disabled%         \addtolength{\@tempdima}{-\baselineskip}%
%disabled%         % move to the calculated vertical possition
%disabled%         \vspace*{\@tempdima}%
%disabled%         % put the thumbs left so the current horizontal possition
%disabled%         \llap{%
%disabled%           % and rotate them
%disabled%           \rotatebox{90}{\colorbox{\chapterthumbboxcolor}{%
%disabled%               \parbox[c][\chapterthumbheight][c]{\chapterthumbwidth}{%
%disabled%                 \centering
%disabled%                 \textcolor{\chapterthumbtextcolor}{%
%disabled%                   \strut\thechapter}\\
%disabled%               }%
%disabled%             }%
%disabled%           }%
%disabled%         }%
%disabled%         % avoid overfull \vbox messages
%disabled%         \vss
%disabled%       }%
%disabled%     }%
%disabled%   \endgroup
%disabled% }
%disabled%  
%disabled% % New command, which works like \lohead but also puts the thumbs (you
%disabled% % cannot use \ihead with this definition but you may change this, if
%disabled% % you use more internal scrpage2 commands)
%disabled% \newcommand*{\loheadwithchapterthumbs}[2][]{%
%disabled%   \lohead[\putchapterthumb#1]{\putchapterthumb#2}%
%disabled% }
%disabled%  
%disabled% % initial use
%disabled% \loheadwithchapterthumbs{}
%disabled% \pagestyle{scrheadings}
%disabled%  
%disabled% }
%disabled% 

%%%% END
%%% Local Variables:
%%% mode: latex
%%% mode: auto-fill
%%% mode: flyspell
%%% eval: (ispell-change-dictionary "en_US")
%%% TeX-master: "../main"
%%% End:
%% vim:foldmethod=expr
%% vim:fde=getline(v\:lnum)=~'^%%%%'?0\:getline(v\:lnum)=~'^%doc.*\ .\\%(sub\\)\\?section{.\\+'?'>1'\:'1':
# command
%doc% in \texttt{main.tex}.  For standard usage it is recommended to stay with the
%doc% default settings.
%doc% 
%doc% 
%% ========================================================================

%doc%
%doc% Some basic microtypographic settings are provided by the
%doc% \texttt{microtype}
%doc% package\footnote{\url{http://ctan.org/pkg/microtype}}. This template
%doc% uses the rather conservative package parameters: \texttt{protrusion=true,factor=900}.
\usepackage[protrusion=true,factor=900]{microtype}

%doc%
%doc% \subsection{French spacing}
%doc%
%doc% \paragraph{Why?} see~\textcite[p.\,28, p.\,30]{Bringhurst1993}: `2.1.4 Use a single word space between sentences.'
%doc%
%doc% \paragraph{How?} see~\textcite[p.\,185]{Eijkhout2008}:\\
%doc% \verb#\frenchspacing  %% Macro to switch off extra space after punctuation.# \\
\frenchspacing  %% Macro to switch off extra space after punctuation.
%doc%
%doc% Note: This setting might be default for \myacro{KOMA} script.
%doc%


%doc%
%doc% \subsection{Font}
%doc% 
%doc% This template is using the Palatino font (package \texttt{mathpazo}) which results
%doc% in a legible document and matching mathematical fonts for printout.
%doc% 
%doc% It is highly recommended that you either stick to the Palatino font or use the
%doc% \LaTeX{} default fonts (by removing the package \texttt{mathpazo}).
%doc% 
%doc% Chosing different fonts is not
%doc% an easy task. Please leave this to people with good knowledge on this subject.
%doc% 
%doc% One valid reason to change the default fonts is when your document is mainly
%doc% read on a computer screen. In this case it is recommended to switch to a font
%doc% \textsf{which is sans-serif like this}. This template contains several alternative
%doc% font packages which can be activated in this file.
%doc% 

% for changing the default font, please go to the next subsection!

%doc%
%doc% \subsection{Text figures}
%doc% 
%doc% \ldots also called old style numbers such as 0123456789. 
%doc% (German: \enquote{Mediäval\-ziffern\footnote{\url{https://secure.wikimedia.org/wikibooks/de/wiki/LaTeX-W\%C3\%B6rterbuch:\_Medi\%C3\%A4valziffern}}})
%doc% \paragraph{Why?} see~\textcite[p.\,44f]{Bringhurst1993}: 
%doc% \begin{quote}
%doc% `3.2.1 If the font includes both text figures and titling figures, use
%doc%  titling figures only with full caps, and text figures in all other
%doc%  circumstances.'
%doc% \end{quote}
%doc% 
%doc% \paragraph{How?} 
%doc% Quoted from Wikibooks\footnote{\url{https://secure.wikimedia.org/wikibooks/en/wiki/LaTeX/Formatting\#Text\_figures\_.28.22old\_style.22\_numerals.29}}:
%doc% \begin{quote}
%doc% Some fonts do not have text figures built in; the textcomp package attempts to
%doc% remedy this by effectively generating text figures from the currently-selected
%doc% font. Put \verb#\usepackage{textcomp}# in your preamble. textcomp also allows you to
%doc% use decimal points, properly formatted dollar signs, etc. within
%doc% \verb#\oldstylenums{}#.
%doc% \end{quote}
%doc% \ldots but proposed \LaTeX{} method does not work out well. Instead use:\\
%doc% \verb#\usepackage{hfoldsty}#  (enables text figures using additional font) or \\
%doc% \verb#\usepackage[sc,osf]{mathpazo}# (switches to Palatino font with small caps and old style figures enabled).
%doc%
%\usepackage{hfoldsty}  %% enables text figures using additional font
%% ... OR use ...
\usepackage[sc,osf]{mathpazo} %% switches to Palatino with small caps and old style figures

%% Font selection from:
%%     http://www.matthiaspospiech.de/latex/vorlagen/allgemein/preambel/fonts/
%% use following lines *instead* of the mathpazo package above:
%% ===== Serif =========================================================
%% for Computer Modern (LaTeX default font), simply remove the mathpazo above
%\usepackage{charter}\linespread{1.05} %% Charter
%\usepackage{bookman}                  %% Bookman (laedt Avant Garde !!)
%\usepackage{newcent}                  %% New Century Schoolbook (laedt Avant Garde !!)
%% ===== Sans Serif ====================================================
%\renewcommand{\familydefault}{\sfdefault}  %% this one in *combination* with the default mathpazo package
%\usepackage{cmbright}                  %% CM-Bright (eigntlich eine Familie)
%\usepackage{tpslifonts}                %% tpslifonts % Font for Slides


%doc% 
%doc% \subsection{\texttt{myacro} --- Abbrevations using \textsc{small caps}}\myinteresting
%doc% \label{sec:myacro}
%doc% 
%doc% \paragraph{Why?} see~\textcite[p.\,45f]{Bringhurst1993}: `3.2.2 For abbrevations and
%doc% acronyms in the midst of normal text, use spaced small caps.'
%doc% 
%doc% \paragraph{How?} Using the predefined macro \verb#\myacro{}# for things like
%doc% \myacro{UNO} or \myacro{UNESCO} using \verb#\myacro{UNO}# or \verb#\myacro{UNESCO}#.
%doc% 
\DeclareRobustCommand{\myacro}[1]{\textsc{\lowercase{#1}}} %%  abbrevations using small caps


%doc% 
%doc% \subsection{Colorized headings and links}
%doc% 
%doc% This document template is able to generate an output that uses colorized
%doc% headings, captions, page numbers, and links. The color named `DispositionColor'
%doc% used in this document is defined near the definition of package \texttt{color}
%doc% in the preamble (see section~\ref{subsec:miscpackages}). The changes required
%doc% for headings, page numbers, and captions are defined here.
%doc% 
%doc% Settings for colored links are handled by the definitions of the
%doc% \texttt{hyperref} package (see section~\ref{sec:pdf}).
%doc% 
\setheadsepline{.4pt}[\color{DispositionColor}]
\renewcommand{\headfont}{\normalfont\sffamily\color{DispositionColor}}
\renewcommand{\pnumfont}{\normalfont\sffamily\color{DispositionColor}}
\addtokomafont{disposition}{\color{DispositionColor}}
\addtokomafont{caption}{\color{DispositionColor}\footnotesize}
\addtokomafont{captionlabel}{\color{DispositionColor}}

%doc% 
%doc% \subsection{No figures or tables below footnotes}
%doc% 
%doc% \LaTeX{} places floating environments below footnotes if \texttt{b}
%doc% (bottom) is used as (default) placement algorithm. This is certainly
%doc% not appealing for most people and is deactivated in this template by
%doc% using the package \texttt{footmisc} with its option \texttt{bottom}.
%doc% 
%% see also: http://www.komascript.de/node/858 (German description)
\usepackage[bottom]{footmisc}

%doc% 
%doc% \subsection{Spacings of list environments}
%doc% 
%doc% By default, \LaTeX{} is using vertical spaces between items of enumerate, 
%doc% itemize and description environments. This is fine for multi-line items.
%doc% Many times, the user does just write single-line items where the larger
%doc% vertical space is inappropriate. The \href{http://ctan.org/pkg/enumitem}{enumitem}
%doc% package provides replacements for the pre-defined list environments and
%doc% offers many options to modify their appearances.
%doc% This template is using the package option for \texttt{noitemsep} which
%doc% mimimizes the vertical space between list items.
%doc% 
\usepackage{enumitem}
\setlist{noitemsep}   %% kills the space between items

%doc% 
%doc% \subsection{\texttt{csquotes} --- Correct quotation marks}\myinteresting
%doc% \label{sub:csquotes}
%doc% 
%doc% \emph{Never} use quotation marks found on your keyboard.
%doc% They end up in strange characters or false looking quotation marks.
%doc% 
%doc% In \LaTeX{} you are able to use typographically correct quotation marks. The package 
%doc% \href{http://www.ctan.org/pkg/csquotes}{\texttt{csquotes}} offers you with 
%doc% \verb#\enquote{foobar}# a command to get correct quotation marks around \enquote{foobar}.
%doc% Please do check the package options in order to modify
%doc% its settings according to the language used\footnote{most of the time in 
%doc% combination with the language set in the options of the \texttt{babel} package}.
%doc% 
%doc% \href{http://www.ctan.org/pkg/csquotes}{\texttt{csquotes}} is also recommended 
%doc% by \texttt{biblatex} (see Section~\ref{sec:references}). 
\usepackage[babel=true,strict=true,english=american,german=guillemets]{csquotes}

%doc% 
%doc% \subsection{Line spread}
%doc% 
%doc% If you have to enlarge the distance between two lines of text, you can
%doc% increase it using the \texttt{\mylinespread} command in \texttt{main.tex}. By default, it is
%doc% deactivated (set to 100~percent). Modify only with caution since it influences the
%doc% page layout and could lead to ugly looking documents.
\linespread{\mylinespread}

%doc% 
%doc% \subsection{Optional: Lines above and below the chapter head}
%doc% 
%doc% This is not quite something typographic but rather a matter of taste.
%doc% \myacro{KOMA} Script offers \href{http://www.komascript.de/node/24}{a method to
%doc% add lines above and below chapter head} which is disabled by
%doc% default. If you want to enable this feature, remove corresponding
%doc% comment characters from the settings.
%doc% 
%% Source: http://www.komascript.de/node/24
%disabled% %% 1st get a new command
%disabled% \newcommand*{\ORIGchapterheadstartvskip}{}%
%disabled% %% 2nd save the original definition to the new command
%disabled% \let\ORIGchapterheadstartvskip=\chapterheadstartvskip
%disabled% %% 3rd redefine the command using the saved original command
%disabled% \renewcommand*{\chapterheadstartvskip}{%
%disabled%   \ORIGchapterheadstartvskip
%disabled%   {%
%disabled%     \setlength{\parskip}{0pt}%
%disabled%     \noindent\color{DispositionColor}\rule[.3\baselineskip]{\linewidth}{1pt}\par
%disabled%   }%
%disabled% }
%disabled% %% see above
%disabled% \newcommand*{\ORIGchapterheadendvskip}{}%
%disabled% \let\ORIGchapterheadendvskip=\chapterheadendvskip
%disabled% \renewcommand*{\chapterheadendvskip}{%
%disabled%   {%
%disabled%     \setlength{\parskip}{0pt}%
%disabled%     \noindent\color{DispositionColor}\rule[.3\baselineskip]{\linewidth}{1pt}\par
%disabled%   }%
%disabled%   \ORIGchapterheadendvskip
%disabled% }

%doc% 
%doc% \subsection{Optional: Chapter thumbs}
%doc% 
%doc% This is not quite something typographic but rather a matter of taste.
%doc% \myacro{KOMA} Script offers \href{http://www.komascript.de/chapterthumbs-example}{a method to
%doc% add chapter thumbs} (in combination with the package \texttt{scrpage2}) which is disabled by
%doc% default. If you want to enable this feature, remove corresponding
%doc% comment characters from the settings.
%doc% 
%disabled% % Safty first
%disabled% \@ifundefined{chapter}{\let\chapter\undefined
%disabled%   \chapter must be defined to use chapter thumbs!}{%
%disabled%  
%disabled% % Two new commands for the width and height of the boxes with the
%disabled% % chapter number at the thumbs (use of commands instead of lengths
%disabled% % for sparing registers)
%disabled% \newcommand*{\chapterthumbwidth}{2em}
%disabled% \newcommand*{\chapterthumbheight}{1em}
%disabled%  
%disabled% % Two new commands for the colors of the box background and the
%disabled% % chapter numbers of the thumbs
%disabled% \newcommand*{\chapterthumbboxcolor}{black}
%disabled% \newcommand*{\chapterthumbtextcolor}{white}
%disabled%  
%disabled% % New command to set a chapter thumb. I'm using a group at this
%disabled% % command, because I'm changing the temporary dimension \@tempdima
%disabled% \newcommand*{\putchapterthumb}{%
%disabled%   \begingroup
%disabled%     \Large
%disabled%     % calculate the horizontal possition of the right paper border
%disabled%     % (I ignore \hoffset, because I interprete \hoffset moves the page
%disabled%     % at the paper e.g. if you are using cropmarks)
%disabled%     \setlength{\@tempdima}{\@oddheadshift}% (internal from scrpage2)
%disabled%     \setlength{\@tempdima}{-\@tempdima}%
%disabled%     \addtolength{\@tempdima}{\paperwidth}%
%disabled%     \addtolength{\@tempdima}{-\oddsidemargin}%
%disabled%     \addtolength{\@tempdima}{-1in}%
%disabled%     % putting the thumbs should not change the horizontal
%disabled%     % possition
%disabled%     \rlap{%
%disabled%       % move to the calculated horizontal possition
%disabled%       \hspace*{\@tempdima}%
%disabled%       % putting the thumbs should not change the vertical
%disabled%       % possition
%disabled%       \vbox to 0pt{%
%disabled%         % calculate the vertical possition of the thumbs (I ignore
%disabled%         % \voffset for the same reasons told above)
%disabled%         \setlength{\@tempdima}{\chapterthumbwidth}%
%disabled%         \multiply\@tempdima by\value{chapter}%
%disabled%         \addtolength{\@tempdima}{-\chapterthumbwidth}%
%disabled%         \addtolength{\@tempdima}{-\baselineskip}%
%disabled%         % move to the calculated vertical possition
%disabled%         \vspace*{\@tempdima}%
%disabled%         % put the thumbs left so the current horizontal possition
%disabled%         \llap{%
%disabled%           % and rotate them
%disabled%           \rotatebox{90}{\colorbox{\chapterthumbboxcolor}{%
%disabled%               \parbox[c][\chapterthumbheight][c]{\chapterthumbwidth}{%
%disabled%                 \centering
%disabled%                 \textcolor{\chapterthumbtextcolor}{%
%disabled%                   \strut\thechapter}\\
%disabled%               }%
%disabled%             }%
%disabled%           }%
%disabled%         }%
%disabled%         % avoid overfull \vbox messages
%disabled%         \vss
%disabled%       }%
%disabled%     }%
%disabled%   \endgroup
%disabled% }
%disabled%  
%disabled% % New command, which works like \lohead but also puts the thumbs (you
%disabled% % cannot use \ihead with this definition but you may change this, if
%disabled% % you use more internal scrpage2 commands)
%disabled% \newcommand*{\loheadwithchapterthumbs}[2][]{%
%disabled%   \lohead[\putchapterthumb#1]{\putchapterthumb#2}%
%disabled% }
%disabled%  
%disabled% % initial use
%disabled% \loheadwithchapterthumbs{}
%disabled% \pagestyle{scrheadings}
%disabled%  
%disabled% }
%disabled% 

%%%% END
%%% Local Variables:
%%% mode: latex
%%% mode: auto-fill
%%% mode: flyspell
%%% eval: (ispell-change-dictionary "en_US")
%%% TeX-master: "../main"
%%% End:
%% vim:foldmethod=expr
%% vim:fde=getline(v\:lnum)=~'^%%%%'?0\:getline(v\:lnum)=~'^%doc.*\ .\\%(sub\\)\\?section{.\\+'?'>1'\:'1':
# command
%doc% in \texttt{main.tex}.  For standard usage it is recommended to stay with the
%doc% default settings.
%doc% 
%doc% 
%% ========================================================================

%doc%
%doc% Some basic microtypographic settings are provided by the
%doc% \texttt{microtype}
%doc% package\footnote{\url{http://ctan.org/pkg/microtype}}. This template
%doc% uses the rather conservative package parameters: \texttt{protrusion=true,factor=900}.
\usepackage[protrusion=true,factor=900]{microtype}

%doc%
%doc% \subsection{French spacing}
%doc%
%doc% \paragraph{Why?} see~\textcite[p.\,28, p.\,30]{Bringhurst1993}: `2.1.4 Use a single word space between sentences.'
%doc%
%doc% \paragraph{How?} see~\textcite[p.\,185]{Eijkhout2008}:\\
%doc% \verb#\frenchspacing  %% Macro to switch off extra space after punctuation.# \\
\frenchspacing  %% Macro to switch off extra space after punctuation.
%doc%
%doc% Note: This setting might be default for \myacro{KOMA} script.
%doc%


%doc%
%doc% \subsection{Font}
%doc% 
%doc% This template is using the Palatino font (package \texttt{mathpazo}) which results
%doc% in a legible document and matching mathematical fonts for printout.
%doc% 
%doc% It is highly recommended that you either stick to the Palatino font or use the
%doc% \LaTeX{} default fonts (by removing the package \texttt{mathpazo}).
%doc% 
%doc% Chosing different fonts is not
%doc% an easy task. Please leave this to people with good knowledge on this subject.
%doc% 
%doc% One valid reason to change the default fonts is when your document is mainly
%doc% read on a computer screen. In this case it is recommended to switch to a font
%doc% \textsf{which is sans-serif like this}. This template contains several alternative
%doc% font packages which can be activated in this file.
%doc% 

% for changing the default font, please go to the next subsection!

%doc%
%doc% \subsection{Text figures}
%doc% 
%doc% \ldots also called old style numbers such as 0123456789. 
%doc% (German: \enquote{Mediäval\-ziffern\footnote{\url{https://secure.wikimedia.org/wikibooks/de/wiki/LaTeX-W\%C3\%B6rterbuch:\_Medi\%C3\%A4valziffern}}})
%doc% \paragraph{Why?} see~\textcite[p.\,44f]{Bringhurst1993}: 
%doc% \begin{quote}
%doc% `3.2.1 If the font includes both text figures and titling figures, use
%doc%  titling figures only with full caps, and text figures in all other
%doc%  circumstances.'
%doc% \end{quote}
%doc% 
%doc% \paragraph{How?} 
%doc% Quoted from Wikibooks\footnote{\url{https://secure.wikimedia.org/wikibooks/en/wiki/LaTeX/Formatting\#Text\_figures\_.28.22old\_style.22\_numerals.29}}:
%doc% \begin{quote}
%doc% Some fonts do not have text figures built in; the textcomp package attempts to
%doc% remedy this by effectively generating text figures from the currently-selected
%doc% font. Put \verb#\usepackage{textcomp}# in your preamble. textcomp also allows you to
%doc% use decimal points, properly formatted dollar signs, etc. within
%doc% \verb#\oldstylenums{}#.
%doc% \end{quote}
%doc% \ldots but proposed \LaTeX{} method does not work out well. Instead use:\\
%doc% \verb#\usepackage{hfoldsty}#  (enables text figures using additional font) or \\
%doc% \verb#\usepackage[sc,osf]{mathpazo}# (switches to Palatino font with small caps and old style figures enabled).
%doc%
%\usepackage{hfoldsty}  %% enables text figures using additional font
%% ... OR use ...
\usepackage[sc,osf]{mathpazo} %% switches to Palatino with small caps and old style figures

%% Font selection from:
%%     http://www.matthiaspospiech.de/latex/vorlagen/allgemein/preambel/fonts/
%% use following lines *instead* of the mathpazo package above:
%% ===== Serif =========================================================
%% for Computer Modern (LaTeX default font), simply remove the mathpazo above
%\usepackage{charter}\linespread{1.05} %% Charter
%\usepackage{bookman}                  %% Bookman (laedt Avant Garde !!)
%\usepackage{newcent}                  %% New Century Schoolbook (laedt Avant Garde !!)
%% ===== Sans Serif ====================================================
%\renewcommand{\familydefault}{\sfdefault}  %% this one in *combination* with the default mathpazo package
%\usepackage{cmbright}                  %% CM-Bright (eigntlich eine Familie)
%\usepackage{tpslifonts}                %% tpslifonts % Font for Slides


%doc% 
%doc% \subsection{\texttt{myacro} --- Abbrevations using \textsc{small caps}}\myinteresting
%doc% \label{sec:myacro}
%doc% 
%doc% \paragraph{Why?} see~\textcite[p.\,45f]{Bringhurst1993}: `3.2.2 For abbrevations and
%doc% acronyms in the midst of normal text, use spaced small caps.'
%doc% 
%doc% \paragraph{How?} Using the predefined macro \verb#\myacro{}# for things like
%doc% \myacro{UNO} or \myacro{UNESCO} using \verb#\myacro{UNO}# or \verb#\myacro{UNESCO}#.
%doc% 
\DeclareRobustCommand{\myacro}[1]{\textsc{\lowercase{#1}}} %%  abbrevations using small caps


%doc% 
%doc% \subsection{Colorized headings and links}
%doc% 
%doc% This document template is able to generate an output that uses colorized
%doc% headings, captions, page numbers, and links. The color named `DispositionColor'
%doc% used in this document is defined near the definition of package \texttt{color}
%doc% in the preamble (see section~\ref{subsec:miscpackages}). The changes required
%doc% for headings, page numbers, and captions are defined here.
%doc% 
%doc% Settings for colored links are handled by the definitions of the
%doc% \texttt{hyperref} package (see section~\ref{sec:pdf}).
%doc% 
\setheadsepline{.4pt}[\color{DispositionColor}]
\renewcommand{\headfont}{\normalfont\sffamily\color{DispositionColor}}
\renewcommand{\pnumfont}{\normalfont\sffamily\color{DispositionColor}}
\addtokomafont{disposition}{\color{DispositionColor}}
\addtokomafont{caption}{\color{DispositionColor}\footnotesize}
\addtokomafont{captionlabel}{\color{DispositionColor}}

%doc% 
%doc% \subsection{No figures or tables below footnotes}
%doc% 
%doc% \LaTeX{} places floating environments below footnotes if \texttt{b}
%doc% (bottom) is used as (default) placement algorithm. This is certainly
%doc% not appealing for most people and is deactivated in this template by
%doc% using the package \texttt{footmisc} with its option \texttt{bottom}.
%doc% 
%% see also: http://www.komascript.de/node/858 (German description)
\usepackage[bottom]{footmisc}

%doc% 
%doc% \subsection{Spacings of list environments}
%doc% 
%doc% By default, \LaTeX{} is using vertical spaces between items of enumerate, 
%doc% itemize and description environments. This is fine for multi-line items.
%doc% Many times, the user does just write single-line items where the larger
%doc% vertical space is inappropriate. The \href{http://ctan.org/pkg/enumitem}{enumitem}
%doc% package provides replacements for the pre-defined list environments and
%doc% offers many options to modify their appearances.
%doc% This template is using the package option for \texttt{noitemsep} which
%doc% mimimizes the vertical space between list items.
%doc% 
\usepackage{enumitem}
\setlist{noitemsep}   %% kills the space between items

%doc% 
%doc% \subsection{\texttt{csquotes} --- Correct quotation marks}\myinteresting
%doc% \label{sub:csquotes}
%doc% 
%doc% \emph{Never} use quotation marks found on your keyboard.
%doc% They end up in strange characters or false looking quotation marks.
%doc% 
%doc% In \LaTeX{} you are able to use typographically correct quotation marks. The package 
%doc% \href{http://www.ctan.org/pkg/csquotes}{\texttt{csquotes}} offers you with 
%doc% \verb#\enquote{foobar}# a command to get correct quotation marks around \enquote{foobar}.
%doc% Please do check the package options in order to modify
%doc% its settings according to the language used\footnote{most of the time in 
%doc% combination with the language set in the options of the \texttt{babel} package}.
%doc% 
%doc% \href{http://www.ctan.org/pkg/csquotes}{\texttt{csquotes}} is also recommended 
%doc% by \texttt{biblatex} (see Section~\ref{sec:references}). 
\usepackage[babel=true,strict=true,english=american,german=guillemets]{csquotes}

%doc% 
%doc% \subsection{Line spread}
%doc% 
%doc% If you have to enlarge the distance between two lines of text, you can
%doc% increase it using the \texttt{\mylinespread} command in \texttt{main.tex}. By default, it is
%doc% deactivated (set to 100~percent). Modify only with caution since it influences the
%doc% page layout and could lead to ugly looking documents.
\linespread{\mylinespread}

%doc% 
%doc% \subsection{Optional: Lines above and below the chapter head}
%doc% 
%doc% This is not quite something typographic but rather a matter of taste.
%doc% \myacro{KOMA} Script offers \href{http://www.komascript.de/node/24}{a method to
%doc% add lines above and below chapter head} which is disabled by
%doc% default. If you want to enable this feature, remove corresponding
%doc% comment characters from the settings.
%doc% 
%% Source: http://www.komascript.de/node/24
%disabled% %% 1st get a new command
%disabled% \newcommand*{\ORIGchapterheadstartvskip}{}%
%disabled% %% 2nd save the original definition to the new command
%disabled% \let\ORIGchapterheadstartvskip=\chapterheadstartvskip
%disabled% %% 3rd redefine the command using the saved original command
%disabled% \renewcommand*{\chapterheadstartvskip}{%
%disabled%   \ORIGchapterheadstartvskip
%disabled%   {%
%disabled%     \setlength{\parskip}{0pt}%
%disabled%     \noindent\color{DispositionColor}\rule[.3\baselineskip]{\linewidth}{1pt}\par
%disabled%   }%
%disabled% }
%disabled% %% see above
%disabled% \newcommand*{\ORIGchapterheadendvskip}{}%
%disabled% \let\ORIGchapterheadendvskip=\chapterheadendvskip
%disabled% \renewcommand*{\chapterheadendvskip}{%
%disabled%   {%
%disabled%     \setlength{\parskip}{0pt}%
%disabled%     \noindent\color{DispositionColor}\rule[.3\baselineskip]{\linewidth}{1pt}\par
%disabled%   }%
%disabled%   \ORIGchapterheadendvskip
%disabled% }

%doc% 
%doc% \subsection{Optional: Chapter thumbs}
%doc% 
%doc% This is not quite something typographic but rather a matter of taste.
%doc% \myacro{KOMA} Script offers \href{http://www.komascript.de/chapterthumbs-example}{a method to
%doc% add chapter thumbs} (in combination with the package \texttt{scrpage2}) which is disabled by
%doc% default. If you want to enable this feature, remove corresponding
%doc% comment characters from the settings.
%doc% 
%disabled% % Safty first
%disabled% \@ifundefined{chapter}{\let\chapter\undefined
%disabled%   \chapter must be defined to use chapter thumbs!}{%
%disabled%  
%disabled% % Two new commands for the width and height of the boxes with the
%disabled% % chapter number at the thumbs (use of commands instead of lengths
%disabled% % for sparing registers)
%disabled% \newcommand*{\chapterthumbwidth}{2em}
%disabled% \newcommand*{\chapterthumbheight}{1em}
%disabled%  
%disabled% % Two new commands for the colors of the box background and the
%disabled% % chapter numbers of the thumbs
%disabled% \newcommand*{\chapterthumbboxcolor}{black}
%disabled% \newcommand*{\chapterthumbtextcolor}{white}
%disabled%  
%disabled% % New command to set a chapter thumb. I'm using a group at this
%disabled% % command, because I'm changing the temporary dimension \@tempdima
%disabled% \newcommand*{\putchapterthumb}{%
%disabled%   \begingroup
%disabled%     \Large
%disabled%     % calculate the horizontal possition of the right paper border
%disabled%     % (I ignore \hoffset, because I interprete \hoffset moves the page
%disabled%     % at the paper e.g. if you are using cropmarks)
%disabled%     \setlength{\@tempdima}{\@oddheadshift}% (internal from scrpage2)
%disabled%     \setlength{\@tempdima}{-\@tempdima}%
%disabled%     \addtolength{\@tempdima}{\paperwidth}%
%disabled%     \addtolength{\@tempdima}{-\oddsidemargin}%
%disabled%     \addtolength{\@tempdima}{-1in}%
%disabled%     % putting the thumbs should not change the horizontal
%disabled%     % possition
%disabled%     \rlap{%
%disabled%       % move to the calculated horizontal possition
%disabled%       \hspace*{\@tempdima}%
%disabled%       % putting the thumbs should not change the vertical
%disabled%       % possition
%disabled%       \vbox to 0pt{%
%disabled%         % calculate the vertical possition of the thumbs (I ignore
%disabled%         % \voffset for the same reasons told above)
%disabled%         \setlength{\@tempdima}{\chapterthumbwidth}%
%disabled%         \multiply\@tempdima by\value{chapter}%
%disabled%         \addtolength{\@tempdima}{-\chapterthumbwidth}%
%disabled%         \addtolength{\@tempdima}{-\baselineskip}%
%disabled%         % move to the calculated vertical possition
%disabled%         \vspace*{\@tempdima}%
%disabled%         % put the thumbs left so the current horizontal possition
%disabled%         \llap{%
%disabled%           % and rotate them
%disabled%           \rotatebox{90}{\colorbox{\chapterthumbboxcolor}{%
%disabled%               \parbox[c][\chapterthumbheight][c]{\chapterthumbwidth}{%
%disabled%                 \centering
%disabled%                 \textcolor{\chapterthumbtextcolor}{%
%disabled%                   \strut\thechapter}\\
%disabled%               }%
%disabled%             }%
%disabled%           }%
%disabled%         }%
%disabled%         % avoid overfull \vbox messages
%disabled%         \vss
%disabled%       }%
%disabled%     }%
%disabled%   \endgroup
%disabled% }
%disabled%  
%disabled% % New command, which works like \lohead but also puts the thumbs (you
%disabled% % cannot use \ihead with this definition but you may change this, if
%disabled% % you use more internal scrpage2 commands)
%disabled% \newcommand*{\loheadwithchapterthumbs}[2][]{%
%disabled%   \lohead[\putchapterthumb#1]{\putchapterthumb#2}%
%disabled% }
%disabled%  
%disabled% % initial use
%disabled% \loheadwithchapterthumbs{}
%disabled% \pagestyle{scrheadings}
%disabled%  
%disabled% }
%disabled% 

%%%% END
%%% Local Variables:
%%% mode: latex
%%% mode: auto-fill
%%% mode: flyspell
%%% eval: (ispell-change-dictionary "en_US")
%%% TeX-master: "../main"
%%% End:
%% vim:foldmethod=expr
%% vim:fde=getline(v\:lnum)=~'^%%%%'?0\:getline(v\:lnum)=~'^%doc.*\ .\\%(sub\\)\\?section{.\\+'?'>1'\:'1':

%doc% %%%% Time-stamp: <2018-03-11 14:31:45 vk>
%%%% === Disclaimer: =======================================================
%% created by
%%
%%      Karl Voit
%%
%% using GNU/Linux, GNU Emacs & LaTeX 2e
%%

%doc%
%doc% \section{\texttt{pdf\_settings.tex} --- Settings related to PDF output}
%doc% \label{sec:pdf}
%doc%
%doc% The file \verb#template/pdf_settings.tex# basically contains the definitions for
%doc% the \href{http://tug.org/applications/hyperref/}{\texttt{hyperref} package}
%doc% including the
%doc% \href{http://www.ctan.org/tex-archive/macros/latex/required/graphics/}{\texttt{graphicx}
%doc% package}. Since these settings should be the last things of any \LaTeX{}
%doc% preamble, they got their own \TeX{} file which is included in \texttt{main.tex}.
%doc%
%doc% \paragraph{What should I do with this file?} The settings in this file are
%doc% important for \myacro{PDF} output and including graphics. Do not exclude the
%doc% related \texttt{input} command in \texttt{main.tex}. But you might want to
%doc% modify some settings after you read the
%doc% \href{http://tug.org/applications/hyperref/}{documentation of the \texttt{hyperref} package}.
%doc%


%% Fix positioning of images in PDF viewers. (disabled by
%% default; see https://github.com/novoid/LaTeX-KOMA-template/issues/4
%% for more information)
%% I do not have time to read about possible side-effect of this
%% package for now.
% \usepackage[hypcap]{caption}

%% Declarations of hyperref should be the last definitions of the preamble:
%% FIXXME: black-and-white-version for printing!

\pdfcompresslevel=9

\usepackage[%
unicode=true, % loads with unicode support
%a4paper=true, %
pdftex, %
backref, %
pagebackref=false, % creates backward references too
bookmarks=true, % generate bookmarks in PDF files
bookmarksopen=false, % when starting with AcrobatReader, the Bookmarkcolumn is opened
pdfpagemode=UseNone, % UseNone, UseOutlines, UseThumbs, FullScreen
plainpages=false, % correct, if pdflatex complains: ``destination with same identifier already exists''
%% colors: https://secure.wikimedia.org/wikibooks/en/wiki/LaTeX/Colors
urlcolor=DispositionColor, %%
linkcolor=DispositionColor, %%
citecolor=DispositionColor, %%
anchorcolor=DispositionColor, %%
colorlinks=\mycolorlinks, % turn on/off colored links (on: better for
                          % on-screen reading; off: better for printout versions)
]{hyperref}

%% all strings need to be loaded after hyperref was loaded with unicode support
%% if not the field is garbled in the output for characters like ČŽĆŠĐ
\hypersetup{
pdftitle={\mytitle}, %
pdfauthor={\myauthor}, %
pdfsubject={\mysubject}, %
pdfcreator={Accomplished with: pdfLaTeX, biber, and hyperref-package. No animals, MS-EULA or BSA-rules were harmed.},
pdfproducer={\myauthor},
pdfkeywords={\mykeywords}
}

%\DeclareGraphicsExtensions{.pdf}

%%%% END
%%% Local Variables:
%%% TeX-master: "../main"
%%% mode: latex
%%% mode: auto-fill
%%% mode: flyspell
%%% eval: (ispell-change-dictionary "en_US")
%%% End:
%% vim:foldmethod=expr
%% vim:fde=getline(v\:lnum)=~'^%%%%'?0\:getline(v\:lnum)=~'^%doc.*\ .\\%(sub\\)\\?section{.\\+'?'>1'\:'1':

%doc% 
%doc% \begin{document}
%doc% %% title page %%
%doc% \title{\mytitle}\subtitle{\mysubject}
%doc% \author{\myauthor}
%doc% \date{\today}
%doc% 
%doc% \maketitle\newpage
%doc% 
%doc% \tableofcontents\newpage
%doc% %%---------------------------------------%%

%doc% 
%doc% \section{How to use this \LaTeX{} document template}
%doc% 
%doc% This \LaTeX{} document template from
%doc% \myLaT\footnote{\url{http://LaTeX.TUGraz.at}} is based on \myacro{KOMA}
%doc% script\footnote{\url{http://komascript.de/}}. You don't need any
%doc% special \myacro{KOMA} knowledge (but it woun't hurt either). It provides an easy to use and
%doc% easy to modify template. All settings are documented and many references to
%doc% additional information sources are given.
%doc% 

%doc% In general, there should not be any reason to modify a file in
%doc% the \texttt{template} folder. \emph{All important settings are
%doc% accessible in the main folder, mostly in the \texttt{main.tex}
%doc% file.} This way, it is easy to get what you need and you can update
%doc% the template independent of the content of the document.
%doc% 
%doc% \newcommand{\myimportant}{%% mark important chapters
%doc%   \marginpar{\vspace{-1em}\rightpointleft}
%doc% }
%doc% \newcommand{\myinteresting}{\marginpar{\vspace{-2em}\PencilLeftDown}}

%doc% 
%doc% The \emph{absolute minimum you should read} is listed below and 
%doc% marked with the hand symbol:\myimportant
%doc% \begin{itemize}
%doc% \item Section~\ref{sec:modifytemplate}: basic configuration of this template.
%doc% \item Section~\ref{sec:howtocompile}: how to generate the \myacro{PDF} file
%doc% \item Section~\ref{sec:references}: using biblatex (instead of bibtex)
%doc% \end{itemize}
%doc% 
%doc% In order to get a perfect resulting document and to get an
%doc% exciting experience with this template, you should definitely consider reading
%doc% following sections which are also marked with the pencil symbol:\myinteresting
%doc% \begin{itemize}
%doc% \item Section~\ref{sec:extending-template}: extend the template with
%doc%   your own usepackages, newcommands, and so forth
%doc% \item Section~\ref{sec:mycommands}: pre-defined commands to make your life easier (e.g., including graphics)
%doc% \item Section~\ref{sec:myacro}: how to do acronyms (like \myacro{ACME}) beautifully
%doc% \item Section~\ref{sub:csquotes}: how to \enquote{quote} text and use parentheses correctly
%doc% \end{itemize}
%doc% 
%doc% The other sections describe all other settings for the sake of completeness. This is
%doc% interesting for learning more about \LaTeX{} and modifying this template to a higher level of detail.

%doc% 
%doc% \newpage
%doc% \subsection{Six Steps to Customize Your Document}\myimportant
%doc% \label{sec:modifytemplate}
%doc%
%doc% This template is optimized to get to the first draft of your thesis
%doc% very quickly. Follow these instructions and you get most of your
%doc% customizing done in a few minutes:
%doc% 
%doc% \newcommand{\myfile}[1]{\texttt{\href{file:#1}{#1}}}
%doc%
%doc% \begin{enumerate}
%doc% \item Modify settings in \texttt{main.tex} to meet your requirements:
%doc%   \begin{itemize}
%doc%   \item Basic settings
%doc%     \begin{itemize}
%doc%     \item Paper size, languages, font size, citation style,
%doc%           title page, and so forth
%doc%     \end{itemize}
%doc%   \item Document metadata
%doc%     \begin{itemize}
%doc%     \item Preferences like \verb+myauthor+, \verb+mytitle+, and so forth
%doc%     \end{itemize}
%doc%   \end{itemize}
%doc% \item Replace \myfile{figures/institution.pdf} with the logo of
%doc% your institution in either \myacro{PDF} or \myacro{PNG}
%doc% format.\footnote{Avoid \myacro{JPEG} format for
%doc% computer-generated (pixcel-oriented) graphics like logos or
%doc% screenshots in general. The \myacro{JEPG} format is for
%doc% photographs \emph{only}.}
%doc% \item Further down in \myfile{main.tex}:
%doc%   \begin{itemize}
%doc%   \item Create your desired structure for the chapters
%doc%         (\verb+%%%% Time-stamp: <2011-12-11 13:25:46 vk>
%% ========================================================================
%%%% Disclaimer
%% ========================================================================
%%
%% created by
%%
%%      Karl Voit
%%
%% using grml GNU/Linux, vim & LaTeX 2e


%% example text content
%% scrartcl and scrreprt starts with section, subsection, subsubsection, ...
%% scrbook starts with part (optional), chapter, section, ...
\chapter{MYSECTION}

This is my text with an example Figure~\ref{fig:example} and example
citation~\cite{Voit2011} or \textcite{Voit2009}. And there is another
citation which is located at the bottom\footcite{tagstore}.

\myfig{TU_Graz_Logo}%% filename in figures folder
      {width=0.1\textwidth,height=0.1\textheight}%% maximum width/height, aspect ratio will be kept
      {Example figure.}%% caption
      {fig:example}%% label

I am sure that you are now able to write your own
document. Always keep in mind: it's the \emph{content} that
matters, not the form. But good typography is able to deliver
the content much better than information set with bad
typography.


%% vim:foldmethod=expr
%% vim:fde=getline(v\:lnum)=~'^%%%%\ .\\+'?'>1'\:'='
%%% Local Variables: 
%%% mode: latex
%%% TeX-master: "main"
%%% End: 
+, \verb+\include{evaluation}+, \ldots)
%doc%   \end{itemize}
%doc% \item Create the \TeX{} files and fill your content into these files you defined in the previous step.
%doc% \item Optionally: Modify \myfile{colophon.tex} to meet your situation.
%doc%   \begin{itemize}
%doc%   \item Please spend a couple of minutes and think about putting your work
%doc%         under an open license\footnote{\url{https://creativecommons.org/licenses/}}
%doc%         in order to follow the spirit of Open Science\footnote{\url{https://en.wikipedia.org/wiki/Open_science}}.
%doc%   \end{itemize}
%doc% \item In case you are using \myacro{GNU} make\footnote{If you
%doc%       don't know, what \myacro{GNU} make is, you are not using it (yet).}: 
%doc%       Put your desired \myacro{PDF} file name in the second line of file
%doc%    \myfile{Makefile}
%doc%    \begin{itemize}
%doc%    \item replace \enquote{Projectname} with your filename
%doc%    \item do not use any file extension like \texttt{.tex} or \texttt{.pdf}
%doc%    \end{itemize}
%doc% \end{enumerate}
%doc% 
%doc% 

%doc% 
%doc% \subsection{License}\myimportant
%doc% \label{sec:license}
%doc% 
%doc% This template is licensed under a Creative Commons Attribution-ShareAlike 3.0 Unported (CC BY-SA 3.0)
%doc%         license\footnote{\url{https://creativecommons.org/licenses/by-sa/3.0/}}:
%doc%     \begin{itemize}
%doc%     \item You can share (to copy, distribute and transmit) this template.
%doc%     \item You can remix (adapt) this template.
%doc%     \item You can make commercial use of the template.
%doc%     \item In case you modify this template and share the derived
%doc%           template: You must attribute the template such that you do not
%doc%           remove (co-)authorship of Karl Voit and you must not remove
%doc%           the URL to the original repository on 
%doc%           github\footnote{\url{https://github.com/novoid/LaTeX-KOMA-template}}.
%doc%     \item If you alter, transform, or build a new template upon 
%doc%           this template, you may distribute the resulting 
%doc%           template only under the same or similar license to this one. 
%doc%     \item There are \emph{no restrictions} of any kind, however, related to the
%doc%           resulting (PDF) document!
%doc%     \item You may remove the colophon (but it's not recommended).
%doc%     \end{itemize}


%doc% 
%doc% 
%doc% \subsection{How to compile this document}\myimportant
%doc% \label{sec:howtocompile}
%doc% 
%doc% I assume that compiling \LaTeX{} documents within your software
%doc% environment is something you have already learned. This template is
%doc% almost like any other \LaTeX{} document except it uses
%doc% state-of-the-art tools for generating things like the list of
%doc% references using biblatex/biber (see
%doc% Section~\ref{sec:references} for details). Unfortunately, some \LaTeX{} editors
%doc% do not support this much better way of working with bibliography
%doc% references yet. This section describes how to compile this template.
%doc% 
%doc% \subsubsection{Compiling Using a \LaTeX{} Editor}
%doc% 
%doc% Please do select \myfile{main.tex} as the \enquote{main project file} or make
%doc% sure to compile/run only \myfile{main.tex} (and not \myfile{introduction.tex}
%doc% or other \TeX{} files of this template).
%doc% 
%doc% Choose \texttt{biber} for generating the references. Modern LaTeX{}
%doc% environments offer this option. Older tools might not be that up to
%doc% date yet.
%doc% 

%doc% \subsubsection{Activating \texttt{biber} in the \LaTeX{} editor TeXworks}
%doc% \label{sec:biberTeXworks}
%doc% 
%doc% The \href{https://www.tug.org/texworks/}{TeXworks} editor is a very
%doc% basic (but fine) \LaTeX{} editor to start with. It is included in
%doc% \href{http://miktex.org/}{MiKTeX} and
%doc% \href{http://miktex.org/portable}{MiKTeX portable} and supports
%doc% \href{https://en.wikipedia.org/wiki/Syntax_highlighting}{syntax
%doc%   highlighting} and
%doc% \href{http://itexmac.sourceforge.net/SyncTeX.html}{SyncTeX} to
%doc% synchronize \myacro{PDF} output and \LaTeX{} source code.
%doc% 
%doc% Unfortunately, TeXworks shipped with MiKTeX does not support compiling
%doc% using \texttt{biber} (biblatex) out of the box. Here is a solution to
%doc% this issue. Go to TeXworks: \texttt{Edit} $\rightarrow$
%doc% \texttt{Preferences~\ldots} $\rightarrow$ \texttt{Typesetting} $\rightarrow$
%doc% \texttt{Processing tools} and add a new entry (using the plus icon):
%doc% 
%doc% \begin{tabbing}
%doc%   Arguments: \= foobar  \kill
%doc%   Name:      \> \verb#pdflatex+biber# \\
%doc%   Program:   \> \emph{find the \texttt{template/pdflatex+biber.bat} file from your disk} \\
%doc%   Arguments: \> \verb+$fullname+ \\
%doc%              \> \verb+$basename+
%doc% \end{tabbing}
%doc% 
%doc% Activate the \enquote{View PDF after running} option.
%doc% 
%doc% Close the preferences dialog and you will now have an additional
%doc% choice in the drop down list for compiling your document. Choose the
%doc% new entry called \verb#pdflatex+biber# and start a happier life with
%doc% \texttt{biber}.
%doc% 
%doc% In case your TeXworks has a German user interface, here the key
%doc% aspects in German as well:
%doc% 
%doc% \begin{otherlanguage}{ngerman}
%doc% 
%doc%   \texttt{Bearbeiten} $\rightarrow$ \texttt{Einstellungen~\ldots} $\rightarrow$
%doc%   \texttt{Textsatz} $\rightarrow$ \texttt{Verarbeitungsprogramme} $\rightarrow$
%doc%   + \emph{(neues Verarbeitungsprogramm)}:
%doc% 
%doc% \begin{tabbing}
%doc%   Befehl/Datei: \= foobar  \kill
%doc%     Name: \> pdflatex+biber \\
%doc%     Befehl/Datei: \> \emph{die \texttt{template/pdflatex+biber.bat} im Laufwerk suchen} \\
%doc%     Argumente: \> \verb+$fullname+ \\
%doc%                \> \verb+$basename+
%doc% \end{tabbing}
%doc% 
%doc% \enquote{PDF nach Beendigung anzeigen} aktivieren.
%doc% 
%doc% \end{otherlanguage}
%doc% 

%doc% \subsubsection{Compiling Using \myacro{GNU} make}
%doc% 
%doc% With \myacro{GNU}
%doc% make\footnote{\url{https://secure.wikimedia.org/wikipedia/en/wiki/Make\_\%28software\%29}}
%doc% it is just simple as that: \texttt{make pdf}
%doc% 
%doc% Several other targets are available. You can check them out by
%doc% executing: \texttt{make help}
%doc% 

%doc% \subsubsection{Compiling in a Text-Shell}
%doc% 
%doc% To generate a document using \texttt{Biber}, you can stick to
%doc% following example:
%doc% \begin{verbatim}
%doc% pdflatex main.tex
%doc% biber main
%doc% pdflatex main.tex
%doc% pdflatex main.tex
%doc% \end{verbatim}



%doc% 
%doc% \subsection{How to get rid of the template documentation}
%doc% 
%doc% Simply remove the files \verb#Template_Documentation.pdf# and 
%doc% \verb#Template_Documentation.tex# (if it exists) in the main folder 
%doc% of this template.
%doc% 
%doc% \subsection{What about modifying or extending the template?}\myinteresting
%doc% \label{sec:extending-template}
%doc% 
%doc% This template provides an easy to start \LaTeX{} document template with sound
%doc% default settings. You can modify each setting any time. It is recommended that
%doc% you are familiar with the documentation of the command whose settings you want
%doc% to modify.
%doc% 
%doc% It is recommended that for \emph{adding} things to the preambel (newcommands,
%doc% setting variables, defining headers, \dots) you should use the file
%doc% \texttt{main.tex}. 
%doc% There are comment lines which help you find the right spot.
%doc% This way you still have the chance to update your \texttt{template}
%doc% folder from the template repository without losing your own added things.
%doc% 
%doc% The following sections describe the settings and commands of this template and
%doc% give a short overview of its features.

%doc% \subsection{How to change the title page}
%doc% 
%doc% This template comes with a variety of title pages. They are located in
%doc% the folder \texttt{template}. You can switch to a specific title
%doc% page by including the corresponding title page file in the file
%doc% \texttt{main.tex}.
%doc% 
%doc% Please note that you may not need to modify any title page document by
%doc% yourself since all relevant information is defined in the file
%doc% \texttt{main.tex}.

%doc%
%doc% \section{\texttt{preamble.tex} --- Main preamble file}
%doc%
%doc% In the file \verb#preamble/preamble.tex# you will find the basic
%doc% definitions related to your document. This template uses the \myacro{KOMA} script
%doc% extension package of \LaTeX{}.
%doc% 
%doc% There are comments added to the \verb#\documentclass{}# definitions. Please
%doc% refer to the great documentation of \myacro{KOMA}\footnote{\texttt{scrguide.pdf} for
%doc% German users} for further details.
%doc% 
%doc% \paragraph{What should I do with this file?} For standard purposes you might
%doc% use the default values it provides. You must not remove its \texttt{include} command
%doc% in \texttt{main.tex} since it contains important definitions. This file contains
%doc% settings which are documented well and can be modified according to your needs.
%doc% It is recommended that you fully understand each setting you modify in order to
%doc% get a good document result. However, you can set basic values in the
%doc% \texttt{main.tex} file: font size, paper size,
%doc% paragraph separation mode, draft mode, binding correction, and whether
%doc% your document will be a one sided document or you are planning to
%doc% create a document which is printed on both, left side and right side.
%doc% 

\documentclass[%
fontsize=\myfontsize,%% size of the main text
paper=\mypapersize,  %% paper format
parskip=\myparskip,  %% vertical space between paragraphs (instead of indenting first par-line)
DIV=calc,            %% calculates a good DIV value for type area; 66 characters/line is great
headinclude=true,    %% is header part of margin space or part of page content?
footinclude=false,   %% is footer part of margin space or part of page content?
open=right,          %% "right" or "left": start new chapter on right or left page
appendixprefix=true, %% adds appendix prefix; only for book-classes with \backmatter
bibliography=totoc,  %% adds the bibliography to table of contents (without number)
draft=\mydraft,      %% if true: included graphics are omitted and black boxes
                     %%          mark overfull boxes in margin space
BCOR=\myBCOR,        %% binding correction (depends on how you bind
                     %% the resulting printout.
\mylaterality        %% oneside: document is not printed on left and right sides, only right side
                     %% twoside: document is printed on left and right sides
]{scrbook}  %% article class of KOMA: "scrartcl", "scrreprt", or "scrbook".
            %% CAUTION: If documentclass will be changed, *many* other things
            %%          change as well like heading structure, ...



% FIXXME: adopting class usage:
% from scrbook -> scrartcl OR scrreport:
% - remove appendixprefix from class options
% - remove \frontmatter \mainmatter \backmatter \appendix from main.tex

% FIXXME: adopting language:
% add or modify language parameter of package »babel« and use language switches described in babel-documentation

%doc% 
%doc% \subsection{\texttt{inputenc}: UTF8 as input charset}
%doc% 
%doc% You are able and should use \myacro{UTF8} character settings for writing these \TeX{}-files.
%doc% 
%\usepackage{ucs}             %% UTF8 as input characters; UCS incompatible to biblatex
\usepackage[utf8]{inputenc} %% UTF8 as input characters
%% Source: http://latex.tugraz.at/latex/tutorial#laden_von_paketen


%doc% 
%doc% \subsection{\texttt{babel}: Language settings}
%doc% 
%doc% The default setting of the language is American. Please change settings for
%doc% additional or alternative languages used in \texttt{main.tex}.
%doc% 
%doc% Please note that the default language of the document is the \emph{last} language
%doc% which is added to the package options.
%doc% 
%doc% To set only parts of your document in a different language as the rest, use for example
%doc% \verb+\foreignlanguage{ngerman}{Beispieltext in deutscher Sprache}+.
%doc% For using foreign language quotes, please refer to the \verb+\foreignquote+,
%doc% \verb+\foreigntextquote+, or \verb+\foreignblockquote+ provided by
%doc% \texttt{csquotes} (see Section~\ref{sub:csquotes}).
%doc% 
\usepackage[\mylanguage]{babel}  %% used languages; default language is *last* language of options

%doc% 
%doc% \subsection{\texttt{scrpage2}: Headers and footers}
%doc% 
%doc% Since this template is based on \myacro{KOMA} script it uses its great \texttt{scrpage2}
%doc% package for defining header and footer information. Please refer to the \myacro{KOMA}
%doc% script documentation how to use this package.
%doc% 
\usepackage{scrpage2} %%  advanced page style using KOMA


%doc% 
%doc% \subsection{References}\myimportant
%doc% \label{sec:references}
%doc% 
%doc% This template is using
%doc% \href{http://www.tex.ac.uk/tex-archive/info/translations/biblatex/de/}{\texttt{biblatex}}
%doc% and \href{http://en.wikipedia.org/wiki/Biber_(LaTeX)}{\texttt{Biber}}
%doc% instead of
%doc% \href{http://en.wikipedia.org/wiki/BibTeX}{\textsc{Bib}\TeX{}}. This has the following
%doc% advantages:
%doc% \begin{itemize}
%doc% \item better documentation
%doc% \item Unicode-support like German umlauts (ö, ä, ü, ß) for references
%doc% \item flexible definition of citation styles
%doc% \item multiple bibliographies e.\,g. for printed and online resources
%doc% \item cleaner reference definition e.\,g. inheriting information from
%doc%   \texttt{Proceedings} to all related \texttt{InProceedings}
%doc% \item modern implementation
%doc% \end{itemize}
%doc% 
%doc% In short, \texttt{biblatex} is able to handle your \texttt{bib}-files
%doc% and offers additional features. To get the most out of
%doc% \texttt{biblatex}, you should read the very good package
%doc% documentation. Be warned: you'll probably never want to change back
%doc% to \textsc{Bib}\TeX{} again.
%doc% 
%doc% Take a look at the files \texttt{references-bibtex.bib} and
%doc% \texttt{references-biblatex.bib}: they contain the three
%doc% references \texttt{tagstore}, \texttt{Voit2009}, and
%doc% \texttt{Voit2011}. 
%doc% The second file is optimized for \texttt{biblatex} and
%doc% takes advantage of some features that are not possible with
%doc% \textsc{Bib}\TeX{}.
%doc% 
%doc% This template is ready to use \texttt{biblatex} with \texttt{Biber} as
%doc% reference compiler. You should make sure that you have installed an up
%doc% to date binary of \texttt{Biber} from its
%doc% homepage\footnote{\url{http://biblatex-biber.sourceforge.net/}}.
%doc% 
%doc% 
%doc% In \texttt{main.tex} you can define several general \texttt{biblatex}
%doc% options: citation style, whether or not multiple occurrences of
%doc% authors are replaced with dashes, or if backward references (from
%doc% references to citations) should be added.
%doc% 
%doc% 
%doc% If you are using the LaTeX{} editor TeXworks, please make sure that
%doc% you have read Section~\ref{sec:biberTeXworks} in order to use
%doc% \texttt{biber}.
%doc% 

%doc% \subsubsection{Example citation commands}
%doc% 
%doc% This section demonstrates some example citations using the style \texttt{authoryear}.
%doc% You can change the citation style in \texttt{main.tex} (\texttt{mybiblatexstyle}).
%doc% 
%doc% \begin{itemize}
%doc% \item cite \cite{Eijkhout2008} and cite \cite{Bringhurst1993, Eijkhout2008}.
%doc% \item citet \citet{Eijkhout2008} and citet \citet{Bringhurst1993, Eijkhout2008}.
%doc% \item autocite \autocite{Eijkhout2008} and autocite \autocite{Bringhurst1993, Eijkhout2008}.
%doc% \item autocites \autocites{Eijkhout2008} and autocites \autocites{Bringhurst1993, Eijkhout2008}.
%doc% \item citeauthor \citeauthor{Eijkhout2008} and citeauthor \citeauthor{Bringhurst1993, Eijkhout2008}.
%doc% \item citetitle \citetitle{Eijkhout2008} and citetitle \citetitle{Bringhurst1993, Eijkhout2008}.
%doc% \item citeyear \citeyear{Eijkhout2008} and citeyear \citeyear{Bringhurst1993, Eijkhout2008}.
%doc% \item textcite \textcite{Eijkhout2008} and textcite \textcite{Bringhurst1993, Eijkhout2008}.
%doc% \item smartcite \smartcite{Eijkhout2008} and smartcite \smartcite{Bringhurst1993, Eijkhout2008}.
%doc% \item footcite \footcite{Eijkhout2008} and footcite \footcite{Bringhurst1993, Eijkhout2008}.
%doc% \item footcite with page \footcite[p.42]{Eijkhout2008} and footcite with page \footcite[compare][p.\,42]{Eijkhout2008}.
%doc% \item fullcite \fullcite{Eijkhout2008} and fullcite \fullcite{Bringhurst1993, Eijkhout2008}.
%doc% \end{itemize}
%doc% 
%doc% Please note that the citation style as well as the bibliography style
%doc% can be changed very easily. Refer to the settings in
%doc% \texttt{main.tex} as well as the very good documentation of \texttt{biblatex}.
%doc% 

%doc% \subsubsection{Using this template with \textsc{Bib}\TeX{}}
%doc% 
%doc% If you do not want to use \texttt{Biber} and \texttt{biblatex}, you
%doc% have to change several things:
%doc% \begin{itemize}
%doc% \item in \verb#preamble/preamble.tex#
%doc%   \begin{itemize}
%doc%   \item remove the usepackage command of \texttt{biblatex}
%doc%   \item remove the \verb#\addbibresource{...}# command
%doc%   \end{itemize}
%doc% \item in \verb#main.tex#
%doc%   \begin{itemize}
%doc%   \item replace \verb=\printbibliography= with the usual
%doc%     \verb=\bibliographystyle{yourstyle}= and \verb=\bibliography{yourbibfile}=
%doc%   \end{itemize}
%doc% \item if you are using \myacro{GNU} \texttt{make}: modify \verb=Makefile=
%doc%   \begin{itemize}
%doc%   \item replace \verb#BIBTEX_CMD = biber# with \verb#BIBTEX_CMD = bibtex#
%doc%   \end{itemize}
%doc% \item Use the reference file \texttt{references-bibtex.bib}
%doc%   instead of \texttt{references-biblatex.bib}
%doc% \end{itemize}
%doc% 
%doc% 
\usepackage[backend=biber, %% using "biber" to compile references (instead of "biblatex")
style=\mybiblatexstyle, %% see biblatex documentation
%style=alphabetic, %% see biblatex documentation
dashed=\mybiblatexdashed, %% do *not* replace recurring reference authors with a dash
backref=\mybiblatexbackref, %% create backlings from references to citations
natbib=true, %% offering natbib-compatible commands
hyperref=true, %% using hyperref-package references
]{biblatex}  %% remove, if using BibTeX instead of biblatex

\addbibresource{\mybiblatexfile} %% remove, if using BibTeX instead of biblatex



%doc% 
%doc% \subsection{Miscellaneous packages} \label{subsec:miscpackages}
%doc% 
%doc% There are several packages included by default. You might want to activate or
%doc% deactivate them according to your requirements:
%doc% 
%doc% \begin{enumerate}

%doc% \item[\texttt{\href{http://www.ctan.org/pkg/graphicx}{%%
%doc% graphicx%%
%doc% }}] 
%doc% The widely used package to use graphical images within a \LaTeX{} document.
\usepackage[pdftex]{graphicx}

%doc% \item[\texttt{\href{https://secure.wikimedia.org/wikibooks/en/wiki/LaTeX/Formatting\#Other\_symbols}{%%
%doc% pifont%%
%doc% }}] 
%doc% For additional special characters available by \verb#\ding{}#
\usepackage{pifont}  %% Sonderzeichen fuer Titelseite \ding{}


%doc% \item[\texttt{\href{http://ctan.org/pkg/ifthen}{%%
%doc% ifthen%%
%doc% }}] 
%doc% For using if/then/else statements for example in macros
\usepackage{ifthen}  %% fuer Wiederholungen usw.

%% pre-define ifthen-boolean variables:
\newboolean{myaddcolophon}
\newboolean{myaddlistoftodos}


%doc% \item[\texttt{\href{http://www.ctan.org/tex-archive/fonts/eurosym}{%%
%doc% eurosym%%
%doc% }}] 
%doc% Using the character for Euro with \verb#\officialeuro{}#
%\usepackage{eurosym}

%doc% \item[\texttt{\href{http://www.ctan.org/tex-archive/help/Catalogue/entries/xspace.html}{%%
%doc% xspace%%
%doc% }}] 
%doc% This package is required for intelligent spacing after commands
\usepackage{xspace}

%doc% \item[\texttt{\href{https://secure.wikimedia.org/wikibooks/en/wiki/LaTeX/Colors}{%%
%doc% xcolor%%
%doc% }}] 
%doc% This package defines basic colors. If you want to get rid of colored links and headings
%doc% please change corresponding value in \texttt{main.tex} to \{0,0,0\}.
\usepackage[usenames,dvipsnames]{xcolor}
\definecolor{DispositionColor}{RGB}{\mydispositioncolor} %% used for links and so forth in screen-version

%doc% \item[\texttt{\href{http://www.ctan.org/pkg/ulem}{%%
%doc% ulem%%
%doc% }}] 
%doc% This package offers strikethrough command \verb+\sout{foobar}+.
\usepackage[normalem]{ulem}

%doc% \item[\texttt{\href{http://www.ctan.org/pkg/framed}{%%
%doc% framed%%
%doc% }}] 
%doc% Create framed, shaded, or differently highlighted regions that can 
%doc% break across pages.  The environments defined are 
%doc% \begin{itemize}
%doc%   \item framed: ordinary frame box (\verb+\fbox+) with edge at margin
%doc%   \item shaded: shaded background (\verb+\colorbox+) bleeding into margin
%doc%   \item snugshade: similar
%doc%   \item leftbar: thick vertical line in left margin
%doc% \end{itemize}
\usepackage{framed}

%doc% \item[\texttt{\href{http://www.ctan.org/pkg/eso-pic}{%%
%doc% eso-pic%%
%doc% }}] 
%doc% For example on title pages you might want to have a logo on the upper right corner of
%doc% the first page (only). The package \texttt{eso-pic} is able to place things on absolute
%doc% and relative positions on the whole page.
\usepackage{eso-pic} %%

%doc% \item[\texttt{\href{http://ctan.org/pkg/enumitem}{%%
%doc% enumitem%%
%doc% }}] 
%doc% This package replaces the built-in definitions for enumerate, itemize and description. 
%doc% With \texttt{enumitem} the user has more control over the layout of those environments.
\usepackage{enumitem} %%

%doc% \item[\texttt{\href{http://www.ctan.org/tex-archive/macros/latex/contrib/todonotes/}{%%
%doc% todonotes%%
%doc% }}] 
%doc% This packages is \emph{very} handy to add notes\footnote{\texttt{todonotes} replaced 
%doc% the \texttt{fixxme}-command which previously was defined in the 
%doc% \texttt{preamble\_mycommands.tex} file.}. Using for example \verb#\todo{check}#
%doc% results in something like this \todo{check} in the document. Do read the
%doc% great package documentation for usage of other very helpful commands such as
%doc% \verb#\missingfigure{}# and \verb#\listoftodos#. The latter one creates an index of all
%doc% open todos which is very useful for getting an overview of open issues.
%doc% The package \texttt{todonotes} require the packages \texttt{ifthen}, \texttt{xkeyval}, \texttt{xcolor}, 
%doc% \texttt{tikz}, \texttt{calc}, and \texttt{graphicx}. Activate
%doc% and configure \verb#\listoftodos# in \texttt{main.tex}.
%\usepackage{todonotes}
\usepackage[\mytodonotesoptions]{todonotes}  %% option "disable" removes all todonotes output from resulting document

%disabled% \item[\texttt{\href{http://www.ctan.org/tex-archive/macros/latex/contrib/blindtext}{%%
%disabled% blindtext%%
%disabled% }}] 
%disabled% This package is used to generate blind text for demonstration purposes.
%disabled% %% This is undocumented due to problems using american english; author informed
%disabled% \usepackage{blindtext}  %% provides commands for blind text:
%disabled% %% \blindtext creates some text,
%disabled% %% \Blindtext creates more text.
%disabled% %% \blinddocument creates a small document with sections, lists...
%disabled% %% \Blinddocument creates a large document with sections, lists...
%% 2012-03-10: vk: author published a corrected version which is able to handle "american english" as well. Did not have time to check new package version for this template here.

%doc% \item[\texttt{\href{http://ctan.org/tex-archive/macros/latex/contrib/units}{%%
%doc% units%%
%doc% }}] 
%doc% For setting correctly typesetted units and nice fractions with \verb+\unit[42]{m}+ and \verb+\unitfrac[100]{km}{h}+.
\usepackage{units}


%doc% \end{enumerate}




%%%% End 
%%% Local Variables:
%%% TeX-master: "../main"
%%% mode: latex
%%% mode: auto-fill
%%% mode: flyspell
%%% eval: (ispell-change-dictionary "en_US")
%%% End:
%% vim:foldmethod=expr
%% vim:fde=getline(v\:lnum)=~'^%%%%'?0\:getline(v\:lnum)=~'^%doc.*\ .\\%(sub\\)\\?section{.\\+'?'>1'\:'1':
