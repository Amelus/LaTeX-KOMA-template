%%%% Time-stamp: <2011-12-11 14:02:27 vk>
%%%% === Disclaimer: =======================================================
%% created by
%%
%%      Karl Voit
%%
%% using grml GNU/Linux, vim & LaTeX 2e
%%
%doc%
%doc% \section{\texttt{typographic\_settings.tex} --- Typographic finetuning}
%doc%
%doc% The settings of file \verb#preamble/typographic_settings.tex# contain
%doc% typographic finetuning related to things mentioned in literature.  The
%doc% settings in this file relates to personal taste and most of all typographic
%doc% experience. 
%doc% 
%doc% \paragraph{What should I do with this file?} You might as well skip the whole
%doc% file by excluding the \verb#%%%% Time-stamp: <2011-07-22 15:45:04 vk>
%%%% === Disclaimer: =======================================================
%% created by
%%
%%      Karl Voit
%%
%% using grml GNU/Linux, vim & LaTeX 2e
%%
%doc%
%doc% \section{\texttt{typographic\_settings.tex} --- Typographic finetuning}
%doc%
%doc% The settings of file \verb#preamble/typographic_settings.tex# contain
%doc% typographic finetuning related to things mentioned in literature.  The
%doc% settings in this file relates to personal taste and most of all typographic
%doc% experience. 
%doc% 
%doc% \paragraph{What should I do with this file?} You might as well skip the whole
%doc% file by excluding the \verb#%%%% Time-stamp: <2011-07-22 15:45:04 vk>
%%%% === Disclaimer: =======================================================
%% created by
%%
%%      Karl Voit
%%
%% using grml GNU/Linux, vim & LaTeX 2e
%%
%doc%
%doc% \section{\texttt{typographic\_settings.tex} --- Typographic finetuning}
%doc%
%doc% The settings of file \verb#preamble/typographic_settings.tex# contain
%doc% typographic finetuning related to things mentioned in literature.  The
%doc% settings in this file relates to personal taste and most of all typographic
%doc% experience. 
%doc% 
%doc% \paragraph{What should I do with this file?} You might as well skip the whole
%doc% file by excluding the \verb#%%%% Time-stamp: <2011-07-22 15:45:04 vk>
%%%% === Disclaimer: =======================================================
%% created by
%%
%%      Karl Voit
%%
%% using grml GNU/Linux, vim & LaTeX 2e
%%
%doc%
%doc% \section{\texttt{typographic\_settings.tex} --- Typographic finetuning}
%doc%
%doc% The settings of file \verb#preamble/typographic_settings.tex# contain
%doc% typographic finetuning related to things mentioned in literature.  The
%doc% settings in this file relates to personal taste and most of all typographic
%doc% experience. 
%doc% 
%doc% \paragraph{What should I do with this file?} You might as well skip the whole
%doc% file by excluding the \verb#\input{preamble/typographic_settings.tex}# command
%doc% in \texttt{main.tex}.  For standard usage it is recommended to stay with the
%doc% default settings.
%doc% 
%doc% 
%doc% \subsection{References related to typographic settings}
%doc% 
%doc% \begin{thebibliography}{9}
%doc% \bibitem[Bringhurst1993]{Bringhurst1993}
%doc%     \textbf{Robert Bringhurst}\\
%doc%     \textit{The Elements of Typographic Style}\\
%doc%     paperback, first edition, 1993
%doc% \bibitem[Eijkhout2008]{Eijkhout2008}
%doc%     \textbf{Victor Eijkhout}\\
%doc%     \textit{\TeX{} by Topic, a \TeX{}nician's Reference}\\
%doc%     document revision 1.2, may 2008\\
%doc%     \url{http://www.eijkhout.net/texbytopic/texbytopic.html}
%doc% \end{thebibliography}
%% ========================================================================


%doc%
%doc% \subsection{French spacing}
%doc%
%doc% \paragraph{Why?} \cite[p 28, p 30]{Bringhurst1993}: `2.1.4 Use a single word space between sentences.'
%doc%
%doc% \paragraph{How?} \cite[p 185]{Eijkhout2008}:\\
%doc% \verb#\frenchspacing  %% Macro to switch off extra space after punctuation.# \\
\frenchspacing  %% Macro to switch off extra space after punctuation.
%doc%
%doc% Note: This setting might be default for \myabk{KOMA} script.
%doc%


%doc%
%doc% \subsection{Text figures}
%doc% 
%doc% \ldots also called old style numbers. 
%doc% (German: Mediävalziffern\footnote{\url{https://secure.wikimedia.org/wikibooks/de/wiki/LaTeX-W\%C3\%B6rterbuch:\_Medi\%C3\%A4valziffern}})
%doc% \paragraph{Why?} \cite[p 44f]{Bringhurst1993}: 
%doc% \begin{quote}
%doc% `3.2.1 If the font includes both text figures and titling figures, use
%doc%  titling figures only with full caps, and text figures in all other
%doc%  circumstances.'
%doc% \end{quote}
%doc% 
%doc% \paragraph{How?} 
%doc% Quoted from Wikibooks\footnote{\url{https://secure.wikimedia.org/wikibooks/en/wiki/LaTeX/Formatting\#Text\_figures\_.28.22old\_style.22\_numerals.29}}:
%doc% \begin{quote}
%doc% Some fonts do not have text figures built in; the textcomp package attempts to
%doc% remedy this by effectively generating text figures from the currently-selected
%doc% font. Put \verb#\usepackage{textcomp}# in your preamble. textcomp also allows you to
%doc% use decimal points, properly formatted dollar signs, etc. within
%doc% \verb#\oldstylenums{}#.
%doc% \end{quote}
%doc% \ldots but proposed \LaTeX{} method does not work out well. Instead use:\\
%doc% \verb#\usepackage{hfoldsty}#  (enables text figures using additional font) or \\
%doc% \verb#\usepackage[sc,osf]{mathpazo}# (switches to Palatino font with small caps and old style figures enabled).
%doc%
%\usepackage{hfoldsty}  %% enables text figures using additional font
%% ... OR use ...
\usepackage[sc,osf]{mathpazo} %% switches to Palatino with small caps and old style figures


%doc% 
%doc% \subsection{Abbrevations using \textsc{small caps}}
%doc% 
%doc% \paragraph{Why?} \cite[p 45f]{Bringhurst1993}: `3.2.2 For abbrevations and
%doc% acronyms in the midst of normal text, use spaced small caps.'
%doc% 
%doc% \paragraph{How?} Using the predefined macro \verb#\myabk{}# for things like
%doc% \myabk{UNO} or \myabk{UNESCO} using \verb#\myabk{UNO}# or \verb#\myabk{UNESCO}#.
%doc% 
\newcommand{\myabk}[1]{%%  abbrevations using small caps
\textsc{\lowercase{#1}}%%
}


%doc% 
%doc% \subsection{Colorized headings and links}
%doc% 
%doc% This document template is able to generate an output that uses colorized
%doc% headings, captions, page numbers, and links. The color named `DispositionColor'
%doc% used in this document is defined near the definition of package \texttt{color}
%doc% in the preamble (see section~\ref{subsec:miscpackages}). The changes required
%doc% for headings, page numbers, and captions are defined here.
%doc% 
%doc% Settings for colored links are handled by the definitions of the
%doc% \texttt{hyperref} package (see section~\ref{sec:pdf}).
%doc% 
\setheadsepline{.4pt}[\color{DispositionColor}]
\renewcommand{\headfont}{\normalfont\sffamily\color{DispositionColor}}
\renewcommand{\pnumfont}{\normalfont\sffamily\color{DispositionColor}}
\addtokomafont{disposition}{\color{DispositionColor}}
\addtokomafont{caption}{\color{DispositionColor}\footnotesize}
\addtokomafont{captionlabel}{\color{DispositionColor}}

%%%% END
%%% Local Variables:
%%% TeX-master: "../main"
%%% End:
%% vim:foldmethod=expr
%% vim:fde=getline(v\:lnum)=~'^%%%%'?0\:getline(v\:lnum)=~'^%doc.*\ .\\%(sub\\)\\?section{.\\+'?'>1'\:'1':
# command
%doc% in \texttt{main.tex}.  For standard usage it is recommended to stay with the
%doc% default settings.
%doc% 
%doc% 
%doc% \subsection{References related to typographic settings}
%doc% 
%doc% \begin{thebibliography}{9}
%doc% \bibitem[Bringhurst1993]{Bringhurst1993}
%doc%     \textbf{Robert Bringhurst}\\
%doc%     \textit{The Elements of Typographic Style}\\
%doc%     paperback, first edition, 1993
%doc% \bibitem[Eijkhout2008]{Eijkhout2008}
%doc%     \textbf{Victor Eijkhout}\\
%doc%     \textit{\TeX{} by Topic, a \TeX{}nician's Reference}\\
%doc%     document revision 1.2, may 2008\\
%doc%     \url{http://www.eijkhout.net/texbytopic/texbytopic.html}
%doc% \end{thebibliography}
%% ========================================================================


%doc%
%doc% \subsection{French spacing}
%doc%
%doc% \paragraph{Why?} \cite[p 28, p 30]{Bringhurst1993}: `2.1.4 Use a single word space between sentences.'
%doc%
%doc% \paragraph{How?} \cite[p 185]{Eijkhout2008}:\\
%doc% \verb#\frenchspacing  %% Macro to switch off extra space after punctuation.# \\
\frenchspacing  %% Macro to switch off extra space after punctuation.
%doc%
%doc% Note: This setting might be default for \myabk{KOMA} script.
%doc%


%doc%
%doc% \subsection{Text figures}
%doc% 
%doc% \ldots also called old style numbers. 
%doc% (German: Mediävalziffern\footnote{\url{https://secure.wikimedia.org/wikibooks/de/wiki/LaTeX-W\%C3\%B6rterbuch:\_Medi\%C3\%A4valziffern}})
%doc% \paragraph{Why?} \cite[p 44f]{Bringhurst1993}: 
%doc% \begin{quote}
%doc% `3.2.1 If the font includes both text figures and titling figures, use
%doc%  titling figures only with full caps, and text figures in all other
%doc%  circumstances.'
%doc% \end{quote}
%doc% 
%doc% \paragraph{How?} 
%doc% Quoted from Wikibooks\footnote{\url{https://secure.wikimedia.org/wikibooks/en/wiki/LaTeX/Formatting\#Text\_figures\_.28.22old\_style.22\_numerals.29}}:
%doc% \begin{quote}
%doc% Some fonts do not have text figures built in; the textcomp package attempts to
%doc% remedy this by effectively generating text figures from the currently-selected
%doc% font. Put \verb#\usepackage{textcomp}# in your preamble. textcomp also allows you to
%doc% use decimal points, properly formatted dollar signs, etc. within
%doc% \verb#\oldstylenums{}#.
%doc% \end{quote}
%doc% \ldots but proposed \LaTeX{} method does not work out well. Instead use:\\
%doc% \verb#\usepackage{hfoldsty}#  (enables text figures using additional font) or \\
%doc% \verb#\usepackage[sc,osf]{mathpazo}# (switches to Palatino font with small caps and old style figures enabled).
%doc%
%\usepackage{hfoldsty}  %% enables text figures using additional font
%% ... OR use ...
\usepackage[sc,osf]{mathpazo} %% switches to Palatino with small caps and old style figures


%doc% 
%doc% \subsection{Abbrevations using \textsc{small caps}}
%doc% 
%doc% \paragraph{Why?} \cite[p 45f]{Bringhurst1993}: `3.2.2 For abbrevations and
%doc% acronyms in the midst of normal text, use spaced small caps.'
%doc% 
%doc% \paragraph{How?} Using the predefined macro \verb#\myabk{}# for things like
%doc% \myabk{UNO} or \myabk{UNESCO} using \verb#\myabk{UNO}# or \verb#\myabk{UNESCO}#.
%doc% 
\newcommand{\myabk}[1]{%%  abbrevations using small caps
\textsc{\lowercase{#1}}%%
}


%doc% 
%doc% \subsection{Colorized headings and links}
%doc% 
%doc% This document template is able to generate an output that uses colorized
%doc% headings, captions, page numbers, and links. The color named `DispositionColor'
%doc% used in this document is defined near the definition of package \texttt{color}
%doc% in the preamble (see section~\ref{subsec:miscpackages}). The changes required
%doc% for headings, page numbers, and captions are defined here.
%doc% 
%doc% Settings for colored links are handled by the definitions of the
%doc% \texttt{hyperref} package (see section~\ref{sec:pdf}).
%doc% 
\setheadsepline{.4pt}[\color{DispositionColor}]
\renewcommand{\headfont}{\normalfont\sffamily\color{DispositionColor}}
\renewcommand{\pnumfont}{\normalfont\sffamily\color{DispositionColor}}
\addtokomafont{disposition}{\color{DispositionColor}}
\addtokomafont{caption}{\color{DispositionColor}\footnotesize}
\addtokomafont{captionlabel}{\color{DispositionColor}}

%%%% END
%%% Local Variables:
%%% TeX-master: "../main"
%%% End:
%% vim:foldmethod=expr
%% vim:fde=getline(v\:lnum)=~'^%%%%'?0\:getline(v\:lnum)=~'^%doc.*\ .\\%(sub\\)\\?section{.\\+'?'>1'\:'1':
# command
%doc% in \texttt{main.tex}.  For standard usage it is recommended to stay with the
%doc% default settings.
%doc% 
%doc% 
%doc% \subsection{References related to typographic settings}
%doc% 
%doc% \begin{thebibliography}{9}
%doc% \bibitem[Bringhurst1993]{Bringhurst1993}
%doc%     \textbf{Robert Bringhurst}\\
%doc%     \textit{The Elements of Typographic Style}\\
%doc%     paperback, first edition, 1993
%doc% \bibitem[Eijkhout2008]{Eijkhout2008}
%doc%     \textbf{Victor Eijkhout}\\
%doc%     \textit{\TeX{} by Topic, a \TeX{}nician's Reference}\\
%doc%     document revision 1.2, may 2008\\
%doc%     \url{http://www.eijkhout.net/texbytopic/texbytopic.html}
%doc% \end{thebibliography}
%% ========================================================================


%doc%
%doc% \subsection{French spacing}
%doc%
%doc% \paragraph{Why?} \cite[p 28, p 30]{Bringhurst1993}: `2.1.4 Use a single word space between sentences.'
%doc%
%doc% \paragraph{How?} \cite[p 185]{Eijkhout2008}:\\
%doc% \verb#\frenchspacing  %% Macro to switch off extra space after punctuation.# \\
\frenchspacing  %% Macro to switch off extra space after punctuation.
%doc%
%doc% Note: This setting might be default for \myabk{KOMA} script.
%doc%


%doc%
%doc% \subsection{Text figures}
%doc% 
%doc% \ldots also called old style numbers. 
%doc% (German: Mediävalziffern\footnote{\url{https://secure.wikimedia.org/wikibooks/de/wiki/LaTeX-W\%C3\%B6rterbuch:\_Medi\%C3\%A4valziffern}})
%doc% \paragraph{Why?} \cite[p 44f]{Bringhurst1993}: 
%doc% \begin{quote}
%doc% `3.2.1 If the font includes both text figures and titling figures, use
%doc%  titling figures only with full caps, and text figures in all other
%doc%  circumstances.'
%doc% \end{quote}
%doc% 
%doc% \paragraph{How?} 
%doc% Quoted from Wikibooks\footnote{\url{https://secure.wikimedia.org/wikibooks/en/wiki/LaTeX/Formatting\#Text\_figures\_.28.22old\_style.22\_numerals.29}}:
%doc% \begin{quote}
%doc% Some fonts do not have text figures built in; the textcomp package attempts to
%doc% remedy this by effectively generating text figures from the currently-selected
%doc% font. Put \verb#\usepackage{textcomp}# in your preamble. textcomp also allows you to
%doc% use decimal points, properly formatted dollar signs, etc. within
%doc% \verb#\oldstylenums{}#.
%doc% \end{quote}
%doc% \ldots but proposed \LaTeX{} method does not work out well. Instead use:\\
%doc% \verb#\usepackage{hfoldsty}#  (enables text figures using additional font) or \\
%doc% \verb#\usepackage[sc,osf]{mathpazo}# (switches to Palatino font with small caps and old style figures enabled).
%doc%
%\usepackage{hfoldsty}  %% enables text figures using additional font
%% ... OR use ...
\usepackage[sc,osf]{mathpazo} %% switches to Palatino with small caps and old style figures


%doc% 
%doc% \subsection{Abbrevations using \textsc{small caps}}
%doc% 
%doc% \paragraph{Why?} \cite[p 45f]{Bringhurst1993}: `3.2.2 For abbrevations and
%doc% acronyms in the midst of normal text, use spaced small caps.'
%doc% 
%doc% \paragraph{How?} Using the predefined macro \verb#\myabk{}# for things like
%doc% \myabk{UNO} or \myabk{UNESCO} using \verb#\myabk{UNO}# or \verb#\myabk{UNESCO}#.
%doc% 
\newcommand{\myabk}[1]{%%  abbrevations using small caps
\textsc{\lowercase{#1}}%%
}


%doc% 
%doc% \subsection{Colorized headings and links}
%doc% 
%doc% This document template is able to generate an output that uses colorized
%doc% headings, captions, page numbers, and links. The color named `DispositionColor'
%doc% used in this document is defined near the definition of package \texttt{color}
%doc% in the preamble (see section~\ref{subsec:miscpackages}). The changes required
%doc% for headings, page numbers, and captions are defined here.
%doc% 
%doc% Settings for colored links are handled by the definitions of the
%doc% \texttt{hyperref} package (see section~\ref{sec:pdf}).
%doc% 
\setheadsepline{.4pt}[\color{DispositionColor}]
\renewcommand{\headfont}{\normalfont\sffamily\color{DispositionColor}}
\renewcommand{\pnumfont}{\normalfont\sffamily\color{DispositionColor}}
\addtokomafont{disposition}{\color{DispositionColor}}
\addtokomafont{caption}{\color{DispositionColor}\footnotesize}
\addtokomafont{captionlabel}{\color{DispositionColor}}

%%%% END
%%% Local Variables:
%%% TeX-master: "../main"
%%% End:
%% vim:foldmethod=expr
%% vim:fde=getline(v\:lnum)=~'^%%%%'?0\:getline(v\:lnum)=~'^%doc.*\ .\\%(sub\\)\\?section{.\\+'?'>1'\:'1':
# command
%doc% in \texttt{main.tex}.  For standard usage it is recommended to stay with the
%doc% default settings.
%doc% 
%doc% 
%doc% \subsection{References related to typographic settings}
%doc% 
%doc% \printbibliography
%doc% 
%% ========================================================================


%doc%
%doc% \subsection{French spacing}
%doc%
%doc% \paragraph{Why?} \cite[p.\,28, p.\,30]{Bringhurst1993}: `2.1.4 Use a single word space between sentences.'
%doc%
%doc% \paragraph{How?} \cite[p.\,185]{Eijkhout2008}:\\
%doc% \verb#\frenchspacing  %% Macro to switch off extra space after punctuation.# \\
\frenchspacing  %% Macro to switch off extra space after punctuation.
%doc%
%doc% Note: This setting might be default for \myabk{KOMA} script.
%doc%


%doc%
%doc% \subsection{Text figures}
%doc% 
%doc% \ldots also called old style numbers. 
%doc% (German: \myquote{Mediävalziffern}\footnote{\url{https://secure.wikimedia.org/wikibooks/de/wiki/LaTeX-W\%C3\%B6rterbuch:\_Medi\%C3\%A4valziffern}})
%doc% \paragraph{Why?} \cite[p.\,44f]{Bringhurst1993}: 
%doc% \begin{quote}
%doc% `3.2.1 If the font includes both text figures and titling figures, use
%doc%  titling figures only with full caps, and text figures in all other
%doc%  circumstances.'
%doc% \end{quote}
%doc% 
%doc% \paragraph{How?} 
%doc% Quoted from Wikibooks\footnote{\url{https://secure.wikimedia.org/wikibooks/en/wiki/LaTeX/Formatting\#Text\_figures\_.28.22old\_style.22\_numerals.29}}:
%doc% \begin{quote}
%doc% Some fonts do not have text figures built in; the textcomp package attempts to
%doc% remedy this by effectively generating text figures from the currently-selected
%doc% font. Put \verb#\usepackage{textcomp}# in your preamble. textcomp also allows you to
%doc% use decimal points, properly formatted dollar signs, etc. within
%doc% \verb#\oldstylenums{}#.
%doc% \end{quote}
%doc% \ldots but proposed \LaTeX{} method does not work out well. Instead use:\\
%doc% \verb#\usepackage{hfoldsty}#  (enables text figures using additional font) or \\
%doc% \verb#\usepackage[sc,osf]{mathpazo}# (switches to Palatino font with small caps and old style figures enabled).
%doc%
%\usepackage{hfoldsty}  %% enables text figures using additional font
%% ... OR use ...
\usepackage[sc,osf]{mathpazo} %% switches to Palatino with small caps and old style figures


%doc% 
%doc% \subsection{Abbrevations using \textsc{small caps}}
%doc% 
%doc% \paragraph{Why?} \cite[p.\,45f]{Bringhurst1993}: `3.2.2 For abbrevations and
%doc% acronyms in the midst of normal text, use spaced small caps.'
%doc% 
%doc% \paragraph{How?} Using the predefined macro \verb#\myabk{}# for things like
%doc% \myabk{UNO} or \myabk{UNESCO} using \verb#\myabk{UNO}# or \verb#\myabk{UNESCO}#.
%doc% 
\newcommand{\myabk}[1]{%%  abbrevations using small caps
\textsc{\lowercase{#1}}%%
}


%doc% 
%doc% \subsection{Colorized headings and links}
%doc% 
%doc% This document template is able to generate an output that uses colorized
%doc% headings, captions, page numbers, and links. The color named `DispositionColor'
%doc% used in this document is defined near the definition of package \texttt{color}
%doc% in the preamble (see section~\ref{subsec:miscpackages}). The changes required
%doc% for headings, page numbers, and captions are defined here.
%doc% 
%doc% Settings for colored links are handled by the definitions of the
%doc% \texttt{hyperref} package (see section~\ref{sec:pdf}).
%doc% 
\setheadsepline{.4pt}[\color{DispositionColor}]
\renewcommand{\headfont}{\normalfont\sffamily\color{DispositionColor}}
\renewcommand{\pnumfont}{\normalfont\sffamily\color{DispositionColor}}
\addtokomafont{disposition}{\color{DispositionColor}}
\addtokomafont{caption}{\color{DispositionColor}\footnotesize}
\addtokomafont{captionlabel}{\color{DispositionColor}}

%doc% 
%doc% \subsection{No figures or tables below footnotes}
%doc% 
%doc% \LaTeX{} places floating environments below footnotes if \texttt{b}
%doc% (bottom) is used as (default) placement algorithm. This is certainly
%doc% not appealing for most people and is deactivated in this template by
%doc% using the package \texttt{footmisc} with its option \texttt{bottom}.
%doc% 
%% see also: http://www.komascript.de/node/858 (German description)
\usepackage[bottom]{footmisc}  %% do not place floats below footnotes

%doc% 
%doc% \subsection{Quotation marks}
%doc% 
%doc% For \texttt{biblatex} it is recommended to include the package
%doc% \texttt{csquotes}. Since Section~\ref{sub:myquote} is providing a
%doc% mechanism to modify quotation characters accordingly, you have to make
%doc% sure that you are using the same quotation characters at both settings.
\usepackage[babel,german=guillemets]{csquotes}

%doc% 
%doc% \subsection{Optional: Lines above and below the chapter head}
%doc% 
%doc% This is not quite something typographic but rather a matter of taste.
%doc% KOMA Script offers \href{http://www.komascript.de/node/24}{a method to
%doc% add lines above and below chapter head} which is disabled by
%doc% default. If you want to enable this feature, remove corresponding
%doc% comment characters from the settings.
%doc% 
%% Source: http://www.komascript.de/node/24
%disabled% %% 1st get a new command
%disabled% \newcommand*{\ORIGchapterheadstartvskip}{}%
%disabled% %% 2nd save the original definition to the new command
%disabled% \let\ORIGchapterheadstartvskip=\chapterheadstartvskip
%disabled% %% 3rd redefine the command using the saved original command
%disabled% \renewcommand*{\chapterheadstartvskip}{%
%disabled%   \ORIGchapterheadstartvskip
%disabled%   {%
%disabled%     \setlength{\parskip}{0pt}%
%disabled%     \noindent\color{DispositionColor}\rule[.3\baselineskip]{\linewidth}{1pt}\par
%disabled%   }%
%disabled% }
%disabled% %% see above
%disabled% \newcommand*{\ORIGchapterheadendvskip}{}%
%disabled% \let\ORIGchapterheadendvskip=\chapterheadendvskip
%disabled% \renewcommand*{\chapterheadendvskip}{%
%disabled%   {%
%disabled%     \setlength{\parskip}{0pt}%
%disabled%     \noindent\color{DispositionColor}\rule[.3\baselineskip]{\linewidth}{1pt}\par
%disabled%   }%
%disabled%   \ORIGchapterheadendvskip
%disabled% }

%doc% 
%doc% \subsection{Optional: Chapter thumbs}
%doc% 
%doc% This is not quite something typographic but rather a matter of taste.
%doc% KOMA Script offers \href{http://www.komascript.de/chapterthumbs-example}{a method to
%doc% add chapter thumbs} (in combination with the package \texttt{scrpage2}) which is disabled by
%doc% default. If you want to enable this feature, remove corresponding
%doc% comment characters from the settings.
%doc% 
%disabled% % Safty first
%disabled% \@ifundefined{chapter}{\let\chapter\undefined
%disabled%   \chapter must be defined to use chapter thumbs!}{%
%disabled%  
%disabled% % Two new commands for the width and height of the boxes with the
%disabled% % chapter number at the thumbs (use of commands instead of lengths
%disabled% % for sparing registers)
%disabled% \newcommand*{\chapterthumbwidth}{2em}
%disabled% \newcommand*{\chapterthumbheight}{1em}
%disabled%  
%disabled% % Two new commands for the colors of the box background and the
%disabled% % chapter numbers of the thumbs
%disabled% \newcommand*{\chapterthumbboxcolor}{black}
%disabled% \newcommand*{\chapterthumbtextcolor}{white}
%disabled%  
%disabled% % New command to set a chapter thumb. I'm using a group at this
%disabled% % command, because I'm changing the temporary dimension \@tempdima
%disabled% \newcommand*{\putchapterthumb}{%
%disabled%   \begingroup
%disabled%     \Large
%disabled%     % calculate the horizontal possition of the right paper border
%disabled%     % (I ignore \hoffset, because I interprete \hoffset moves the page
%disabled%     % at the paper e.g. if you are using cropmarks)
%disabled%     \setlength{\@tempdima}{\@oddheadshift}% (internal from scrpage2)
%disabled%     \setlength{\@tempdima}{-\@tempdima}%
%disabled%     \addtolength{\@tempdima}{\paperwidth}%
%disabled%     \addtolength{\@tempdima}{-\oddsidemargin}%
%disabled%     \addtolength{\@tempdima}{-1in}%
%disabled%     % putting the thumbs should not change the horizontal
%disabled%     % possition
%disabled%     \rlap{%
%disabled%       % move to the calculated horizontal possition
%disabled%       \hspace*{\@tempdima}%
%disabled%       % putting the thumbs should not change the vertical
%disabled%       % possition
%disabled%       \vbox to 0pt{%
%disabled%         % calculate the vertical possition of the thumbs (I ignore
%disabled%         % \voffset for the same reasons told above)
%disabled%         \setlength{\@tempdima}{\chapterthumbwidth}%
%disabled%         \multiply\@tempdima by\value{chapter}%
%disabled%         \addtolength{\@tempdima}{-\chapterthumbwidth}%
%disabled%         \addtolength{\@tempdima}{-\baselineskip}%
%disabled%         % move to the calculated vertical possition
%disabled%         \vspace*{\@tempdima}%
%disabled%         % put the thumbs left so the current horizontal possition
%disabled%         \llap{%
%disabled%           % and rotate them
%disabled%           \rotatebox{90}{\colorbox{\chapterthumbboxcolor}{%
%disabled%               \parbox[c][\chapterthumbheight][c]{\chapterthumbwidth}{%
%disabled%                 \centering
%disabled%                 \textcolor{\chapterthumbtextcolor}{%
%disabled%                   \strut\thechapter}\\
%disabled%               }%
%disabled%             }%
%disabled%           }%
%disabled%         }%
%disabled%         % avoid overfull \vbox messages
%disabled%         \vss
%disabled%       }%
%disabled%     }%
%disabled%   \endgroup
%disabled% }
%disabled%  
%disabled% % New command, which works like \lohead but also puts the thumbs (you
%disabled% % cannot use \ihead with this definition but you may change this, if
%disabled% % you use more internal scrpage2 commands)
%disabled% \newcommand*{\loheadwithchapterthumbs}[2][]{%
%disabled%   \lohead[\putchapterthumb#1]{\putchapterthumb#2}%
%disabled% }
%disabled%  
%disabled% % initial use
%disabled% \loheadwithchapterthumbs{}
%disabled% \pagestyle{scrheadings}
%disabled%  
%disabled% }
%disabled% 

%%%% END
%%% Local Variables:
%%% TeX-master: "../main"
%%% End:
%% vim:foldmethod=expr
%% vim:fde=getline(v\:lnum)=~'^%%%%'?0\:getline(v\:lnum)=~'^%doc.*\ .\\%(sub\\)\\?section{.\\+'?'>1'\:'1':
