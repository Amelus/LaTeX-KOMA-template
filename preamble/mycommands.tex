%% Time-stamp: <2011-11-10 13:22:52 vk>
%%%% === Disclaimer: =======================================================
%% created by
%%
%%      Karl Voit
%%
%% using grml GNU/Linux, vim & LaTeX 2e
%%

%doc%
%doc% \section{\texttt{mycommands.tex} --- various definitions}
%doc%
%doc% In file \verb#preamble/mycommands.tex# many useful commands are being
%doc% defined. 
%doc% 
%doc% \paragraph{What should I do with this file?} Please take a look at its 
%doc% content to get the most out of your document.
%doc% 

%doc% 
%doc% One of the best advantages of \LaTeX{} compared to \myabk{WYSIWYG} software products is
%doc% the possibility to define and use macros within text. This empowers the user to
%doc% a great extend.  Many things can be defined using \verb#\newcommand{}# and
%doc% automates repeating tasks. It its recommended to use macros not only for
%doc% repetitive tasks but also for separating form from content such as \myabk{CSS}
%doc% does for \myabk{XHTML}. Think of including graphics in your document: after
%doc% writing your book, you might want to change all captions to the upper side of
%doc% each figure. In this case you either have to modify all
%doc% \texttt{includegraphics} commands or you were clever enough to define something
%doc% like \verb#\myfig#\footnote{See below for a detailed description}. Using a
%doc% macro for including graphics enables you to modify the position caption on only
%doc% \emph{one} place: at the definition of the macro.
%doc% 
%doc% Following section describes some macros that came with this document template
%doc% from \myLaT and you are welcome to modify or extend them or to create
%doc% your own macros!
%doc% 

%doc% 
%doc% \subsection{\texttt{myfig} --- including graphics made easy}
%doc% 
%doc% The classic: you can easily add graphics to you document with \verb#\myfig#:
%doc% \begin{verbatim}
%doc%  \myfig{flower}%% filename w/o extension in directory "figures"
%doc%        {0.7\textwidth}%% maximum width/height, aspect ratio will be kept
%doc%        {This flower was photographed at my home town in 2010}%% caption
%doc%        {fig:flower}%% label
%doc% \end{verbatim}
%doc% 
\newcommand{\myfig}[4]{
%% example:
% \myfig{}%% filename in figures folder
%       {width=0.5\textwidth,height=0.5\textheight}%% maximum width/height, aspect ratio will be kept
%       {}%% caption
%       {}%% label
\begin{figure}%[htp]
  \begin{center}
     \includegraphics[keepaspectratio,#2]{figures/#1}
     \caption{#3}
     \label{#4} %% NOTE: always label *after* caption!
  \end{center}
\end{figure}
}


%doc% 
%doc% \subsection{\texttt{myclone} --- repeat things!}
%doc% 
%doc% Using \verb#\myclone[42]{foobar}# results the text `foobar' printed 42 times.
%doc% But you can not only repeat text output with \texttt{myclone}. 
%doc%
%doc% Default argument
%doc% for the optional parameter `number of times' (like `42' in the example above) 
%doc% is set to two.
%doc% 
%% \myclone[x]{text}
\newcounter{testcnt}
\newcommand{\myclone}[2][2]{%
  \setcounter{myclonecnt}{#1}%
  \whiledo{\value{myclonecnt}>0}{#2\addtocounter{myclonecnt}{-1}}%
}

%% Info: http://en.wikibooks.org/wiki/LaTeX/Internationalization#German
%doc% 
%doc% \subsection{\texttt{myquote} --- correct quotation marks}
%doc% 
%doc% You should \emph{never} use quotation marks found on your keyboard
%doc% since they end up in strange characters or false quotation marks.
%doc% 
%doc% In \LaTeX{} you have to use different quotation marks. With 
%doc% \verb#\myquote{foobar}# you can get correct quotation marks around `foobar'.
%doc% Please do check the definition of this function in order to modify
%doc% its settings according to your language and quotation marks.
%doc%
%% \myquote{text}
\newcommand{\myquote}[1]{%
  %% please do remove the first comment character which
  %% are in front of the command you want to use:
  % \glqq{}#1\grqq{} %% results in something like ,,#1''
  % \glq{}#1\grq{}   %% results in something like ,#1'
   \frqq{}#1\flqq{} %% results in something like >>#1<< (for *german* texts)
  % \frq{}#1\flq{}   %% results in something like >#1< (for *german* texts)
  % \flqq{}#1\frqq{} %% results in something like <<#1>> (for *non* german texts)
  % \flq{}#1\frq{}   %% results in something like <#1> (for *non* german texts)
}

%%%% End 
%%% Local Variables:
%%% TeX-master: "../main"
%%% End:
%% vim:foldmethod=expr
%% vim:fde=getline(v\:lnum)=~'^%%%%'?0\:getline(v\:lnum)=~'^%doc.*\ .\\%(sub\\)\\?section{.\\+'?'>1'\:'1':
