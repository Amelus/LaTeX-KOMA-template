%% Time-stamp: <2011-07-24 18:02:12 vk>
%%%% === Disclaimer: =======================================================
%% created by
%%
%%      Karl Voit
%%
%% using grml GNU/Linux, vim & LaTeX 2e
%%
%doc% 
%doc% \section{How to use this \LaTeX{} document template}
%doc% 
%doc% This \LaTeX{} document template from
%doc% \myLaT\footnote{\url{http://LaTeX.TUGraz.at}} is based on \myabk{KOMA}
%doc% script\footnote{\url{http://komascript.de/}}. It provides an easy to use and
%doc% easy to modify template. All settings are documented and many references to
%doc% additional information sources are given.
%doc% 
%doc% \subsection{Modify this template for your requirements}
%doc%
%doc% \newcommand{\myfile}[1]{\texttt{\href{file:#1}{#1}}}
%doc%
%doc% \begin{enumerate}
%doc% \item Put your desired \myabk{PDF} file name in the second line of file
%doc%    \myfile{Makefile}
%doc%    \begin{itemize}
%doc%    \item replace \myquote{Projectname} with your filename
%doc%    \item do not use any file extension like \texttt{.tex} or \texttt{.pdf}
%doc%    \end{itemize}
%doc% \item OPTIONAL: Modify files in the folder \texttt{preamble} if necessary
%doc%   \begin{itemize}
%doc%   \item following sections should give you an idea
%doc%     what to do and why
%doc%   \end{itemize}
%doc% \item Modify \myfile{userdata.tex}:
%doc%   \begin{itemize}
%doc%   \item \verb+\myauthor+, \verb+\mytitle+, and so forth
%doc%   \end{itemize}
%doc% \item Modify \myfile{main.tex}:
%doc%   \begin{itemize}
%doc%   \item your desired general document structure
%doc%   \end{itemize}
%doc% \item replace file or fill your content in \myfile{introduction.tex}
%doc%   \begin{itemize}
%doc%   \item you can rename \myfile{introduction.tex} but then you have to
%doc%     modify its include command in \myfile{main.tex} too
%doc%   \end{itemize}
%doc% \item OPTIONAL: create further tex-files (similar to \myfile{introduction.tex}) for each
%doc%    chapter of your document
%doc%    \begin{itemize}
%doc%    \item include them according to the example of \texttt{introduction}
%doc%      in \myfile{main.tex}
%doc%    \end{itemize}
%doc% \item generate your document
%doc%   \begin{itemize}
%doc%   \item with a \LaTeX{} editor:
%doc%     \begin{itemize}
%doc%     \item select \myfile{main.tex} as the \myquote{main project file} or make
%doc%       sure to compile/run only \myfile{main.tex} (and not \myfile{introduction.tex}
%doc%       or similar)
%doc%     \end{itemize}
%doc%   \item OR with \myabk{GNU} make: run \texttt{make pdf}
%doc%   \item OR with plain command line:
%doc%     \begin{itemize}
%doc%     \item run \texttt{pdflatex main.tex} (invoke twice!)
%doc%     \item if you are using
%doc%       \href{http://en.wikipedia.org/wiki/BibTeX}{BibTeX}, start \texttt{bibtex
%doc%       main} followed by \texttt{pdflatex main.tex}
%doc%     \end{itemize}
%doc%   \end{itemize}
%doc% \end{enumerate}
%doc% 
%doc% 
%doc% \subsection{How to compile this document}
%doc% 
%doc% \subsubsection{\textsc{GNU}/Linux, \textsc{OS~X}, \textsc{UNIX}, Cygwin}
%doc% 
%doc% If your system provides \myabk{GNU}
%doc% make\footnote{\url{https://secure.wikimedia.org/wikipedia/en/wiki/Make\_\%28software\%29}},
%doc% it is very easy to compile this document:
%doc% 
%doc% \begin{verbatim}
%doc% make pdf
%doc% \end{verbatim}
%doc% 
%doc% You can get a list of all other commands provided by the Makefile by invoking
%doc% \texttt{make help}.
%doc% 
%doc% If you do not have GNU make, you can compile main.tex within your \LaTeX{} editor.
%doc% 
%doc% 
%doc% \subsubsection{All other systems including Microsoft Windows}
%doc% 
%doc% If your system does not provide \myabk{GNU} make or you do not want to use \myabk{GNU} make,
%doc% you can compile this document using the usual method with pdf\LaTeX{}:
%doc% 
%doc% \begin{verbatim}
%doc% pdflatex main.tex
%doc% pdflatex main.tex
%doc% \end{verbatim}
%doc% 
%doc% If you are using \textsc{Bib}\TeX{} you have to add its commands such as:
%doc% 
%doc% \begin{verbatim}
%doc% pdflatex main.tex
%doc% bibtex main
%doc% pdflatex main.tex
%doc% pdflatex main.tex
%doc% \end{verbatim}
%doc% 
%doc% Addidional commands are required for packages like \texttt{makeindex} and so
%doc% forth.
%doc% 
%doc% \subsection{How to get rid of the template documentation}
%doc% 
%doc% Simply remove the files \verb#Template_Documentation.pdf# and 
%doc% \verb#Template_Documentation.tex# (if it exists) in the main folder 
%doc% of this template.
%doc% 
%doc% \subsection{What about modifying the template?}
%doc% 
%doc% This template provides an easy to start \LaTeX{} document template with sound
%doc% default settings. You can modify each setting any time. It is recommended that
%doc% you are familiar with the documentation of the command whose settings you want
%doc% to modify.
%doc% 
%doc% The following sections describe the settings and commands of this template and
%doc% gives a short overview of its features.

%doc%
%doc% \section{\texttt{preamble.tex} --- Main preamble file}
%doc%
%doc% In file \verb#preamble/preamble.tex# you will find the basic
%doc% definitions related to your document. This template uses the \myabk{KOMA} script
%doc% extension package of \LaTeX{}.
%doc% 
%doc% There are comments added to the \verb#\documentclass{}# definitions. Please
%doc% refer to the great documentation of \myabk{KOMA}\footnote{\texttt{scrguide.pdf} for
%doc% German users} for further details.
%doc% 
%doc% \paragraph{What should I do with this file?} For standard purposes you might
%doc% use the default values it provides. You must not remove its \texttt{include} command
%doc% in \texttt{main.tex} since it contains important definitions. This file contains
%doc% settings which are documented well an can be modified according to your needs.
%doc% It is recommended that you fully understand each setting you modify in order to
%doc% get a good document result.
%doc% 

\documentclass[%
12pt,%%  size of the main text
a4paper,%%  paper format
parskip=half,%%  vertical space between paragraphs (instead of indenting first par-line)
oneside,%%  document is not printed on left and right sides, only right side
%twoside,%%  document is printed on left and right sides
headinclude,%%  FIXXME
footinclude=false,%%  FIXXME
openright%%  FIXXME
]{scrartcl}%% article class of KOMA: FIXXME
%]{scrreprt}%% article class of KOMA: FIXXME
%]{scrbook}%% article class of KOMA: FIXXME


%doc% 
%doc% \subsection{UTF8 as input charset}
%doc% 
%doc% You are able and should use \myabk{UTF8} character settings for writing these \TeX{}-files.
%doc% 
\usepackage{ucs}             %% UTF8 as input characters
\usepackage[utf8x]{inputenc} %% UTF8 as input characters
%% Source: http://latex.tugraz.at/latex/tutorial#laden_von_paketen


%doc% 
%doc% \subsection{Language settings}
%doc% 
%doc% The default setting of the language is American. Please change settings for
%doc% additional or alternative languages used.
%doc% 
\usepackage[american]{babel}  %% American English
%\usepackage[ngerman]{babel}  %% German


%doc% 
%doc% \subsection{Headers and footers}
%doc% 
%doc% Since this template is based on \myabk{KOMA} script it uses its great \texttt{scrpage2}
%doc% package for defining header and footer information. Please refer to the \myabk{KOMA}
%doc% script documentation how to use this package.
%doc% 
\usepackage{scrpage2} %%  advanced page style using KOMA


%doc% 
%doc% \subsection{Miscellaneous packages} \label{subsec:miscpackages}
%doc% 
%doc% There are several packages included by default. You might want to activate or
%doc% deactivate them according to your requirements:
%doc% 
%doc% \begin{enumerate}


%doc% \item[\texttt{\href{https://secure.wikimedia.org/wikibooks/en/wiki/LaTeX/Formatting\#Other\_symbols}{%%
%doc% pifont%%
%doc% }}] 
%doc% For additional special characters available by \verb#\ding{}#
%\usepackage{pifont}  %% Sonderzeichen fuer Titelseite \ding{}

%doc% \item[\texttt{\href{http://ctan.org/pkg/ifthen}{%%
%doc% ifthen%%
%doc% }}] 
%doc% For using if/then/else statements for example in macros
\usepackage{ifthen}  %% fuer Wiederholungen usw.

%doc% \item[\texttt{\href{http://www.ctan.org/tex-archive/fonts/eurosym}{%%
%doc% eurosym%%
%doc% }}] 
%doc% Using the character for Euro with \verb#\officialeuro{}#
%\usepackage{eurosym}

%doc% \item[\texttt{\href{http://www.ctan.org/tex-archive/help/Catalogue/entries/xspace.html}{%%
%doc% xspace%%
%doc% }}] 
%doc% This package is required for intelligent spacing after commands
\usepackage{xspace}

%doc% \item[\texttt{\href{https://secure.wikimedia.org/wikibooks/en/wiki/LaTeX/Colors}{%%
%doc% color%%
%doc% }}] 
%doc% This package defines basic colors
\usepackage[usenames,dvipsnames]{color}
\definecolor{DispositionColor}{RGB}{30,103,182} %% used for links and so forth in screen-version


%% This is undocumented due to problems using american english:
%\usepackage{blindtext}  %% provides commands for blind text:
%% \blindtext creates some text,
%% \Blindtext creates more text.
%% \blinddocument creates a small document with sections, lists...
%% \Blinddocument creates a large document with sections, lists...
%% Source:
%% http://www.ctan.org/tex-archive/macros/latex/contrib/blindtext/


%doc% \end{enumerate}


%%%% End 
%%% Local Variables:
%%% TeX-master: "../main"
%%% End:
%% vim:foldmethod=expr
%% vim:fde=getline(v\:lnum)=~'^%%%%'?0\:getline(v\:lnum)=~'^%doc.*\ .\\%(sub\\)\\?section{.\\+'?'>1'\:'1':
